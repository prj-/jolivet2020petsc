% SIAM Article Template
\documentclass[draft,review,onefignum,onetabnum]{siamart190516}

% Information that is shared between the article and the supplement
% (title and author information, macros, packages, etc.) goes into
% ex_shared.tex. If there is no supplement, this file can be included
% directly.

% SIAM Shared Information Template
% This is information that is shared between the main document and any
% supplement. If no supplement is required, then this information can
% be included directly in the main document.


% Packages and macros go here
\usepackage{lipsum}
\usepackage{amsfonts}
\usepackage{graphicx}
\usepackage{epstopdf}
\usepackage{algorithmic}
\ifpdf
  \DeclareGraphicsExtensions{.eps,.pdf,.png,.jpg}
\else
  \DeclareGraphicsExtensions{.eps}
\fi

% Add a serial/Oxford comma by default.
\newcommand{\creflastconjunction}{, and~}

% Used for creating new theorem and remark environments
\newsiamremark{remark}{Remark}
\newsiamremark{hypothesis}{Hypothesis}
\crefname{hypothesis}{Hypothesis}{Hypotheses}
\newsiamthm{claim}{Claim}

% Sets running headers as well as PDF title and authors
\headers{KSPHPDDM and PCHPDDM}{P. Jolivet and S. Zampini}

% Title. If the supplement option is on, then "Supplementary Material"
% is automatically inserted before the title.
\title{KSPHPDDM and PCHPDDM: extending PETS\MakeLowercase{c} with robust Overlapping Schwarz Methods and advanced Krylov methods
%\title{KSPHPDDM and PCHPDDM: Two Classes of Adaptive Krylov and Overlapping Schwarz Methods in PETS\MakeLowercase{c}
\thanks{Submitted to the editors DATE.
%\funding{This work was funded by the Fog Research Institute under contract no.~FRI-454.}
}}

% Authors: full names plus addresses.
\author{Pierre Jolivet\thanks{CNRS, IRIT--ENSEEIHT, Toulouse, France 
  (\email{pierre.jolivet@enseeiht.fr}).}
\and Stefano Zampini\thanks{King Abdullah University of Science and Technology, Thuwal, Saudi Arabia 
  (\email{stefano.zampini@kaust.edu.sa}).}}

\usepackage{amsopn}
\DeclareMathOperator{\diag}{diag}


%%% Local Variables: 
%%% mode:latex
%%% TeX-master: "ex_article"
%%% End: 


% Optional PDF information
\ifpdf
\hypersetup{
  pdftitle={KSPHPDDM and PCHPDDM: Two Classes of Adaptive Krylov and Overlapping Schwarz Methods in PETSc},
  pdfauthor={P. Jolivet and S. Zampini}
}
\fi

% The next statement enables references to information in the
% supplement. See the xr-hyperref package for details.

% \externaldocument{supplement}

% FundRef data to be entered by SIAM
%<funding-group specific-use="FundRef">
%<award-group>
%<funding-source>
%<named-content content-type="funder-name"> 
%</named-content> 
%<named-content content-type="funder-identifier"> 
%</named-content>
%</funding-source>
%<award-id> </award-id>
%</award-group>
%</funding-group>
\newcommand\col{:}
\newcommand{\pk}[1]{{\color{gray!90!black} #1}}
\usepackage{stmaryrd}
\usepackage{pgfplotstable}
\pgfplotsset{compat=newest}
\pgfplotsset{colormap={paraview}{rgb(0cm)=(0.278431,0.278431,0.858824) rgb(0.1428571429cm)=(0,0,0.360784) rgb(0.2857142857cm)=(0,1,1) rgb(0.4285714286cm)=(0,0.501961,0) rgb(0.5714285714cm)=(1,1,0) rgb(0.7142857143cm)=(1,0.380392,0) rgb(0.8571428571cm)=(0.419608,0,0) rgb(1cm)=(0.878431,0.301961,0.301961)}}
\usepackage{fancyvrb}
\usepackage{subfig}
\usepackage{upquote,textcomp}
\newcommand*{\fvtextcolor}[2]{\textcolor{#1}{#2}}
\definecolor{last}{RGB}{44,123,182}
\definecolor{blueaccent}{RGB}{215,25,28}
\definecolor{myyellow}{RGB}{102,102,17}
\definecolor{myred}{RGB}{197,65,0}
\definecolor{myblue}{RGB}{0,0,235}
\definecolor{myaqua}{RGB}{9,156,156}
\definecolor{mygreen}{RGB}{63,126,28}
\definecolor{paraviewred}{rgb}{0.278431,0.278431,0.858824}
\definecolor{paraviewblue}{rgb}{0.878431,0.301961,0.301961}
\definecolor{darkgray}{RGB}{253,174,97}
\definecolor{mediumgray}{RGB}{255,255,191}
\usepackage[outline]{contour}
\usepackage{tikz}
\usepackage[binary-units=true]{siunitx}
\usetikzlibrary{decorations.text,arrows.meta,calc,plotmarks,shadows,spy,fixedpointarithmetic,patterns,intersections,arrows.meta,shapes,decorations.text,decorations.pathmorphing,backgrounds,fit,positioning,shapes.symbols,chains,3d,calc,external}
\pgfdeclarelayer{background}
\pgfdeclarelayer{foreground}
\pgfsetlayers{background,main,foreground}
\long\def\ifnodedefined#1#2#3{%
    \@ifundefined{pgf@sh@ns@#1}{#3}{#2}%
}
\makeatother
\usepackage{amsmath,etoolbox}
\AtBeginEnvironment{pmatrix}{\setlength{\arraycolsep}{2pt}}
\pgfmathsetmacro{\dof}{4}
\pgfmathsetmacro{\overlap}{1}
\begin{document}

\maketitle

% REQUIRED
\begin{abstract}
  Contemporary applications in computational science and engineering often
    require the solution of linear systems which may be of different sizes,
    shapes, and structures. The goal of this paper is to explain how two
    libraries, PETSc and HPDDM, have been interfaced together in order to offer
    to end-users adaptive Krylov and overlapping Schwarz methods.  The
    flexibility of the implementation %, which for example may deal with various matrix formats, 
    are showcased and explained.  Diverse examples are
    displayed, ranging from fracture mechanics to linear stability analysis, and
    the sets of options are thoroughly documented, so that they may be readily
    tried on other applications and reproduce results.
\end{abstract}

% REQUIRED
\begin{keywords}
  Krylov methods, domain decomposition preconditioners, distributed computing
\end{keywords}

% REQUIRED
\begin{AMS}
  35Q68, 65Y05
\end{AMS}

\section{Introduction}
Implementing and maintaining robust and scalable applications is highly relevant
in the field of computational science and engineering. Some tools exist to help
end-users achieve this goal. One the one hand, PETSc \cite{balay1997cient}, the
Portable and Extensible Toolkit for Scientific computation, is a
well-established software from the community. It may be used to efficiently
discretize partial differential equations, or solve algebraic linear or
nonlinear systems of equations. It comes with many built-in features such as
advanced preconditioning or tailored matrix formats. Also, it can % be used to compile and 
interact with third-party libraries, such as
\emph{hypre}~\cite{falgout2002hypre} for linear solvers, TetGen~\cite{si2013tet}
for mesh generation, and so on. These appealing features convinced computational
scientists to use PETSc as one of the discretization and/or algebraic backend in
many different higher-level projects, such as finite element or finite volume
packages. On the other hand, HPDDM is a much smaller project purely focused on
domain decomposition preconditioners~\cite{jolivet2013scalable} % , a small subclass of preconditioners available in PETSc as well, 
and advanced iterative
methods~\cite{jolivet2016block}. Such methods deal efficiently with linear
systems when multiple right-hand sides are available simultaneously, and are
then referred to as block iterative methods, or when there is a recurrence of
varying coefficient matrices and right-hand sides, and are then referred to as
recycling iterative methods.
% Htool ?
In this paper, the integration of both aspects of HPDDM, its advanced iterative
methods and its domain decomposition preconditioners, into PETSc is explained.
The design principles of the interface are showcased, with % alongside
minimalist and easy-to-run examples. The paper is divided in two main parts:
\cref{sec:KSPHPDDM} which introduces \pk{KSPHPDDM}, and \cref{sec:PCHPDDM} which
introduces \pk{PCHPDDM}. Eventually, concluding remarks may be found
\cref{sec:conclusion}. All PETSc keywords are highlighted in \pk{gray}, these
are public and documented at \url{https://www.mcs.anl.gov/petsc/petsc-master/docs}.
\section{KSPHPDDM}\label{sec:KSPHPDDM}
  \subsection{Related work}
Krylov subspace methods are widely used in numerical linear algebra for solving
systems of equations. Some of their strengths are memory efficiency and the fact
that they are mostly format agnostic. Indeed, they mainly rely on matrix--vector
operations such as multiplications or transpose multiplications. For
challenging high-dimensional systems, finding appropriate preconditioners is
mandatory to make these methods robust. %, but this goes beyond the scope of this section.
In PETSc, Krylov methods belong to the \pk{KSP} base class. Some of the
most used types are \pk{KSPGMRES}, \pk{KSPCG}, or \pk{KSPBCGS}, which respectively implement
the generalized minimal residual method~\cite{saad1986gmres}~(GMRES), conjugate
gradient~\cite{hestenes1952methods}~(CG), or biconjugate gradient stabilized method~\cite{van1992bi}~(BCGS). % Some of these methods are also
% available in other linear algebra backends such as~\emph{hypre} or DUNE~\cite{bastian2010generic}.

Because these methods may need long-term recurrences to mitigate round-off
errors during the generation of Krylov subspaces, it is common to introduce a
restart parameter in order to control their memory consumption and the amount of
global communication, e.g., when orthonormalizing a candidate basis vector. Such
restarts may hinder convergence of iterative methods, and even introduce
convergence plateaus. Recycling techniques have been introduced to attenuate
these effects. In PETSc, the Loose GMRES~\cite{baker2005technique}~(LGMRES) and the
Deflated GMRES~\cite{erhel1996restarted,wakam2013158}~(DGMRES), implemented as
\pk{KSPLGMRES} and \pk{KSPDGMRES}, are available. Neither handle variable
preconditioning, and the latter does not support complex arithmetic. Also note
that when solving a sequence of linear systems $A_i x_i = b_i, i = 1, 2,
\ldots$, it is not trivial to extend the theory of these methods to perform
recycling throughout both the restarts and the different systems. The
generalized conjugate residual method with inner orthogonalization and deflated
restarting~\cite{parks2006recycling}~(GCRODR) does not suffer from this
limitation.

% Recycling put aside, 
Another important aspect of Krylov methods is their ability
to deal with multiple right-hand sides simultaneously. Block Krylov methods~\cite{gutknecht2006block} are
designed for solving linear systems $A X = B$, where
$X$ and $B$ are tall-and-skinny matrices with $k \geq 1$ columns. These methods are
less frequently implemented in general purpose libraries because they require
somehow more involved kernels such as matrix--matrix multiplication, instead of
matrix--vector. Still, systems with multiple right-hand sides are ubiquitous,
e.g., in tomography~\cite{tournier2016micro} or data
analytics~\cite{KALANTZIS2018136}. % Of course, it is possible to cast a system with multiple right-hand sides into a sequence of systems with the same coefficient matrix but changing right-hand sides $A x_i = b_i, i = 1,2,\ldots,k$. However, 
Block Krylov methods, while having higher arithmetic
intensities, generate larger subspaces and typically converge in fewer iterates.
There is none implemented in PETSc. It is nevertheless possible to solve
systems with multiple right-hand sides using a specialized type of matrices,
but with a very limited choice of preconditioners, cf.~\cref{sec:matkaij}.
In the context of optimization, it is also
frequent to solve a sequence of linear systems $A_i X_i = B_i, i = 1, 2,
\ldots$

Trilinos~\cite{heroux2005overview} is another well-known software for scientific
computing. Through its Belos package~\cite{bavier2012amesos2}, it provides most
of the advanced iterative methods aforementioned. Although having PETSc and
Trilinos interoperate is possible, there is currently no \pk{KSP} interface to Belos. % This limits its composability inside complete PETSc applications.
Furthermore,
as displayed in~\cref{sec:interface-ksp}, some Krylov methods implemented in
HPDDM are not available in Belos. In the following section, the details of the
interface between HPDDM Krylov methods and PETSc matrix and preconditioner
classes will be explained. It is assumed that the right-hand sides, and
consequently the solutions, are always dense vectors or matrices.
  \subsection{Interfacing HPDDM Krylov methods in PETSc\label{sec:interface-ksp}}\
\pk{KSPHPDDM}, the interface between HPDDM Krylov methods and PETSc, is automatically
registered since PETSc version 3.12 when configuring PETSc with the extra
flag -{}-download-hpddm. It is then possible to select the corresponding \pk{KSP}
using the command line option -ksp\_type hpddm, or the routine \pk{KSPSetType}(ksp,
\pk{KSPHPDDM}).% and its Fortran equivalent.
When \pk{KSPHPDDM} is being used, it is then possible to select between the following
Krylov methods:
\begin{itemize}
    \item pseudo-block GMRES or flexible GMRES~\cite{saad1993flexible};
    \item pseudo-block CG or flexible CG~\cite{notay2000flexible};
    \item pseudo-block GCRODR or flexible GCRODR~\cite{carvalho2011flexible};
    \item block GMRES or flexible GMRES~\cite{calandra2012flexible};
    \item block CG~\cite{o1980block};
    \item breakdown-free block CG~\cite{ji2017breakdown};
    \item block GCRODR or flexible GCRODR.
\end{itemize}
In pseudo-block variants, multiple similar operations like matrix--vector
products are fused to achieve higher arithmetic intensity, or to decrease the
number of global synchronizations, e.g., reductions for scalar products. But they are mathematically equivalent to
their ``standard'' counterpart. These methods may be selected using the additional
options -ksp\_hpddm\_type (gmres$|$cg$|$gcrodr$|$bgmres$|$bcg$|$bfbcg$|$bgcrodr$|$preonly) and
-ksp\_hpddm\_variant (left$|$right$|$flexible) to select whether preconditioning
is applied on the left, on the right, or variable. It is also possible to set
this option through the more common PETSc option -ksp\_pc\_side (left$|$right). 
Note that the value of the latter option is useless for BCG or BFBCG, and HPDDM CG only handles left and
variable preconditioning. According to the standard PETSc notation, ksp($A$, $P_B$)
defines a \pk{KSP} used to solve a system $Ax=b$ using a preconditioner $P_B$ built
using an operator $B$ which is usually $A$ but not necessarily. If the
type of ksp is \pk{KSPHPDDM}, the solution phase, invoked by \pk{KSPSolve}(ksp, $b$, $x$), will be carried out by HPDDM. To do
so, HPDDM will repeatedly call the following PETSc routines:
\begin{itemize}
    \item \pk{MatMult}($A$, $x$, $y$) for $y = Ax$;
    \item \pk{MatMatMult}($A$, $X$, {\scriptsize \pk{MAT\_REUSE\_MATRIX}}, {\scriptsize
        \pk{PETSC\_DEFAULT}}, $Y$)\footnote{or \pk{MatProductNumeric}($Y$) with PETSc 3.14.0 and above} for $Y =
        AX$;
    \item \pk{PCApply}($P_B$, $y$, $x$) for $x = P_By$.
\end{itemize}
Currently, PETSc does not have a general routine for applying $P_B$ to a
tall-and-skinny dense matrix $Y$. In HPDDM, a specialized routine has been implemented
to compute $X = P_BY$ if the preconditioner is of one of the following type:
\begin{itemize}
    \item exact $LU$ or Cholesky factorization, i.e., $P_B = B^{-1}$, \pk{PCLU} or
        \pk{PCCHOLESKY};
    \item incomplete $LU$ or Cholesky factorization, \pk{PCILU} or \pk{PCICC};
    \item block Jacobi with a single subdomain per process and exact or
        incomplete subdomain solvers, \pk{PCBJACOBI};
    \item overlapping Schwarz methods with a single subdomain per process and
        exact or incomplete subdomain solvers, \pk{PCASM}.
\end{itemize}
Indeed, in these cases, it is possible to get access to a factored matrix $F$
and call \pk{MatMatSolve}($F$, $\tilde{Y}$, $\tilde{X}$). With the two first sets of
preconditioners, $\tilde{Y} = Y$ and $\tilde{X} = X$. With block Jacobi,
$\tilde{Y}$, resp. $\tilde{X}$, represents only the part of the matrix $Y$,
resp. $X$, local to a given process. With overlapping Schwarz methods, there are
furthermore overlapping ``ghost'' unknowns, interested readers are referred to
monographs on practical aspects of domain decomposition
methods~\cite{smith2004domain,dolean2014ddm}. For all other types of
preconditioners, HPDDM will loop over the columns of $Y$ and apply $P_B$
successively to get all columns of $X$.

The rest of the operations are then performed directly inside HPDDM. For
example, in all Krylov methods using the Arnoldi process, it is mandatory to orthonormalize
candidate basis vectors. It would be impractical to try to reuse PETSc routines
like \pk{KSPGMRESModifiedGramSchmidtOrthogonalization} for two reasons: 1) it does
not handle the orthonormalization of multiple vectors at once which is needed in
pseudo-block methods 2) HPDDM may be used without PETSc as a stand-alone
library. To give some more technical details to curious readers, in HPDDM, $QR$
factorizations of tall-and-skinny dense matrices, which are obviously needed for
block Krylov methods but also for recycling Krylov methods are computed using
the CholQR algorithm~\cite{stathopoulos2002block}, or the block modified or classical
Gram--Schmidt method. This may be adjusted with the option -ksp\_hpddm\_qr (cholqr$|$mgs$|$cgs).
For Hessenberg matrices generated by the Arnoldi
process, it is common to update their $QR$ decomposition using Givens
rotations~\cite{saad2003iterative}. For block Hessenberg matrices, Householder reflectors are
used instead~\cite{gutknecht2008updating}.

To monitor convergence of a \pk{KSP} object, calls to the PETSc \pk{KSPMonitor} routine
were added inside HPDDM Krylov methods. 
    \subsubsection{User-defined deflation space} Recycling Krylov methods try to
extract information at the end of each cycle or when convergence is reached by
solving small standard or generalized dense eigenproblems. Then, during a
subsequent solve, this information is reused appropriately. Two new routines were
added in PETSc to either get the information generated by such recycling
methods, or set the information in case users know what is a good initial
deflation subspace. These routines are respectively
\pk{KSPHPDDMGetDeflationSpace} and \pk{KSPHPDDMSetDeflationSpace}. The subspaces
are once again stored as distributed
dense tall-and-skinny matrices.
    \subsubsection{MatKAIJ and systems with multiple right-hand sides\label{sec:matkaij}}
There is a wide range of available matrix types in PETSc. One of them is
\pk{MatKAIJ}, for which it is assumed that an operator may be written as $I \otimes S
+ A \otimes T$, with $S$ and $T$ elements of $\mathbb{K}^{p\times q}$ and
$A$ a standard sparse matrix of dimension $n$. If $p=q=k$, $S = \lambda I$, and $T =
\mu I$, then solving $(I \otimes S + A \otimes T) x = b$ may be cast into a
system with multiple right-hand sides $(\lambda I + \mu A)X = B$ where $B$ is
the row-major representation of $b$, i.e., $b_{k \cdot i + j} = B_{i, j},
\forall (i, j) \in \llbracket 1;n\rrbracket \times \llbracket 1;k\rrbracket$.
On the one hand, due to the structure of the operator, it is challenging to
define appropriate preconditioners for $(I \otimes S + A \otimes T)$: only
\pk{PCPBJACOBI} is currently implemented for \pk{MatKAIJ}. On the other hand,
it may be simpler to efficiently approximate the inverse of $A$. In
\pk{KSPHPDDM}, the type of the operators is checked and if it is \pk{MatKAIJ},
the single right-hand side system is automatically cast into a system with
multiple right-hand sides instead. The standard function \pk{KSPSolve} takes as
input argument a \pk{KSP} and two \pk{Vec}. A new function
\pk{KSPMatSolve} was added in PETSc 3.14.0, which takes as input argument
a \pk{KSP} and two dense \pk{Mat}, for solving true
multiple right-hand side systems. If the input \pk{KSP} is not of type
\pk{KSPHPDDM}, the solution phase is done in a column by column fashion.
Otherwise, it is possible to use any \pk{KSPHPDDM} (pseudo-)block Krylov methods.
Furthermore, a function \pk{KSPSetMatSolveBlockSize} is also provided to decompose
a single large block of column vectors into multiple sub-blocks.
\pk{KSPMatSolve} is then called repeatedly until all sub-blocks are traversed.
This was inspired by MUMPS~\cite{amestoy2001fully} option
ICNTL(27)\footnote{\url{http://mumps.enseeiht.fr/doc/userguide_5.3.1.pdf}, section 6.1}.
  \subsection{Applications and numerical results}
    \subsubsection{Reproducibility of the results from Parks et
    al.~\cite{parks2006recycling}}
Alongside the paper introducing GCRODR, a MATLAB implementation was provided and
is since then
available\footnote{\url{https://www.sandia.gov/~mlparks/GCRODR.zip}}. It comes
with a sequence of ten ``linear systems from a finite element fracture mechanics
problem constructed by Philippe H.\ Geubelle and Spandan Maiti." The goal of
this paragraph is to explain how the results have been reproduced in PETSc. % Thanks to \pk{KSPHPDDM}, it is quite straightforward. First, the sequence of MATLAB systems had to be converted to PETSc \pk{Mat} and \pk{Vec}. Then, it is just a matter of calling \pk{KSPSolve} in sequence.
The dataset and the source code are available at
\url{https://gitlab.com/petsc/datafiles/-/tree/master/matrices/hpddm/GCRODR}
and
\url{https://www.mcs.anl.gov/petsc/petsc-master/src/ksp/ksp/tutorials/ex75.c.html}
respectively. In order to check the correctness of the \pk{KSPHPDDM} interface,
the following tests are performed:
\begin{itemize}
    \item in MATLAB, unpreconditioned GCRODR(40, 20), $ICC(0)$ left-preconditioned
        GCRODR(40, 20), and Jacobi right-preconditioned GCRODR(40, 20);
    \item in PETSc with \pk{KSPGMRES}, unpreconditioned unrestarted GMRES($\infty$);
    \item in PETSc with \pk{KSPHPDDM}, all of the above.
\end{itemize}
All PETSc tests are run on four MPI
processes. They can be launched using the following command line. \\[-4pt]
\begin{Verbatim}[fontsize=\footnotesize,frame=single,framerule=0.1mm,commandchars=&\[\]]
&fvtextcolor[mygreen][$] &fvtextcolor[myyellow][mpirun] -n 4 ./ex75 -ksp_converged_reason -pc_type none -ksp_rtol 1e-10             &fvtextcolor[myred][\]
  -ksp_type hpddm -ksp_hpddm_type gcrodr -ksp_gmres_restart 40 -ksp_hpddm_recycle 20 &fvtextcolor[myred][\] 
  -load_dir &fvtextcolor[myyellow][${DATAFILESPATH}]/matrices/hpddm/GCRODR
&fvtextcolor[mygreen][$] &fvtextcolor[myyellow][mpirun] -n 4 ./ex75 -ksp_converged_reason -sub_pc_type icc -ksp_rtol 1e-10          &fvtextcolor[myred][\]
  -ksp_type hpddm -ksp_hpddm_type gcrodr -ksp_gmres_restart 40 -ksp_hpddm_recycle 20 &fvtextcolor[myred][\] 
  -load_dir &fvtextcolor[myyellow][${DATAFILESPATH}]/matrices/hpddm/GCRODR
&fvtextcolor[mygreen][$] &fvtextcolor[myyellow][mpirun] -n 4 ./ex75 -ksp_converged_reason -pc_type jacobi -ksp_rtol 1e-10           &fvtextcolor[myred][\]
  -ksp_type hpddm -ksp_hpddm_type gcrodr -ksp_gmres_restart 40 -ksp_hpddm_recycle 20 &fvtextcolor[myred][\] 
  -ksp_pc_side right -load_dir &fvtextcolor[myyellow][${DATAFILESPATH}]/matrices/hpddm/GCRODR
&fvtextcolor[mygreen][$] &fvtextcolor[myyellow][mpirun] -n 4 ./ex75 -ksp_converged_reason -pc_type none -ksp_rtol 1e-10             &fvtextcolor[myred][\]
  -ksp_type hpddm -ksp_gmres_restart 500                                             &fvtextcolor[myred][\] 
  -load_dir &fvtextcolor[myyellow][${DATAFILESPATH}]/matrices/hpddm/GCRODR
\end{Verbatim}
The various number of iterations needed to reach the convergence criterion as
originally set to $10^{-10}$ are gathered in~\cref{fig:comparison}. 
Except for the right-preconditioned GCRODR, these tests are the same as the
ones from the original GCRODR paper~\cite{parks2006recycling}. Furthermore,
\pk{PCICC} and ichol from MATLAB are not consistent.  Apart from
the tests \ref{pgfplots:r2} and \ref{pgfplots:r3} using $ICC(0)$ and \ref{pgfplots:r4} and \ref{pgfplots:r5}
using the Jacobi method, one may notice that \ref{pgfplots:r0} and
\ref{pgfplots:r1} (resp. \ref{pgfplots:r6} and \ref{pgfplots:r7}) are almost a
perfect match and reproduce partially Figure 4.2 from Parks et
al.~\cite{parks2006recycling}.
\pgfplotstableread{
sys none-matlab icc-matlab jacobi-matlab none-hpddm icc-hpddm jacobi-hpddm icc-gmres none-gmres icc-petsc none-petsc
401 498         293        457           497        131       460          127       438        127       438
402 228         125        187           231        53        184          127       449        127       449
403 209         113        182           206        47        177          127       450        127       450
404 199         112        175           198        47        174          127       450        127       450
405 198         110        174           198        47        172          127       451        127       451
406 198         111        174           199        47        174          127       452        127       452
407 206         114        180           206        49        179          126       458        126       458
408 207         115        181           208        49        182          128       458        135       458
409 207         110        181           207        49        181          128       458        128       458
410 206         110        179           206        49        179          128       457        128       457
}\loadedtable
    \begin{figure}
\centering \begin{tikzpicture}
    \begin{axis}[xlabel=System index,ylabel={\# of iterations},xmin=401,xmax=410,ymin=0,ymajorgrids,
        legend columns=1,enlarge x limits=0.05,
        legend style={font=\small,at={(1.02,0.25)}, legend cell align=left,
        anchor=south west}]
\addplot[very thick,myred]                       table[x expr=\thisrowno{0},y expr=\thisrowno{1}, col sep=space] {\loadedtable};\label{pgfplots:r0}
\addplot[very thick,mark=*,only marks,myred]     table[x expr=\thisrowno{0},y expr=\thisrowno{4}, col sep=space] {\loadedtable};\label{pgfplots:r1}
\addplot[very thick,myaqua]                      table[x expr=\thisrowno{0},y expr=\thisrowno{2}, col sep=space] {\loadedtable};\label{pgfplots:r2}
\addplot[very thick,mark=pentagon*,only marks,myaqua]    table[x expr=\thisrowno{0},y expr=\thisrowno{5}, col sep=space] {\loadedtable};\label{pgfplots:r3}
\addplot[very thick,myyellow]                    table[x expr=\thisrowno{0},y expr=\thisrowno{3}, col sep=space] {\loadedtable};\label{pgfplots:r4}
\addplot[very thick,mark=square*,only marks,myyellow]  table[x expr=\thisrowno{0},y expr=\thisrowno{6}, col sep=space] {\loadedtable};\label{pgfplots:r5}
\addplot[very thick,last]                      table[x expr=\thisrowno{0},y expr=\thisrowno{10}, col sep=space]{\loadedtable};\label{pgfplots:r6}
\addplot[very thick,mark=triangle*,only marks,last]    table[x expr=\thisrowno{0},y expr=\thisrowno{8}, col sep=space] {\loadedtable};\label{pgfplots:r7}
\legend{{MATLAB unpreconditioned GCRODR},{\pk{KSPHPDDM}},% unpreconditioned GCRODR(40, 20)},
        {MATLAB $ICC(0)$-preconditioned GCRODR},{\pk{KSPHPDDM}},% $ICC(0)$-preconditioned GCRODR(40, 20)},
        {MATLAB Jacobi-preconditioned GCRODR},{\pk{KSPHPDDM}},% Jacobi-preconditioned GCRODR(40, 20)},
        {Unpreconditioned \pk{KSPGMRES}},{\pk{KSPHPDDM}}}% unpreconditioned GMRES($\infty$)}}
\end{axis}
\end{tikzpicture}
        \caption{\# of iterations needed to converge for various
        configurations as originally tested by Parks et
        al.~\cite{parks2006recycling}. Note that \pk{PCICC} and
        \textnormal{ichol} from MATLAB are not consistent, hence the discrepancy
        between \ref{pgfplots:r2} and \ref{pgfplots:r3}. All other \pk{KPSHPDDM}
        numbers of iterations match with the respective MATLAB or PETSc
        reference implementation.\label{fig:comparison}}
    \end{figure}
    \subsubsection{Linear stability analysis}
SLEPc~\cite{hernandez2005ssf}, the Scalable Library for Eigenvalue Problem
computations, is built on top of PETSc. Thus, it is possible to reuse and
compose \pk{KSP} or preconditioners for solving eigenproblems efficiently, e.g.,
through spectral transformations. Recently, it has been observed that in the
field of computational fluid dynamics, the role of recycling can be quite
significant when performing linear stability analysis~\cite{moulin2018al}. 
The following generalized eigenvalue problem is considered next:
\begin{equation}\label{eq:lsa}
    J(q_b)x = \lambda \begin{bmatrix}M & 0 \\ 0 & 0 \end{bmatrix} x,
\end{equation}
where $J$ is the Jacobian of the incompressible steady-state Navier--Stokes equation
evaluated using a given base flow $q_b = \begin{bmatrix}u_b & p_b\end{bmatrix}^T$ and $M$ the discretization of the mass matrix
on the space of velocities. There are many ways to compute the eigenvalues
of~\cref{eq:lsa} near a complex-valued shift $\sigma$, but emphasis will be put
in this paragraph on Krylov--Schur methods~\cite{Stewart2002}. These rely on the successive
solution of linear systems such as:
\begin{equation}\label{eq:ks}
    \left(J(q_b) - \sigma \begin{bmatrix}M&0\\0&0\end{bmatrix}\right)x_i = b_i, i = 1,2,\ldots
\end{equation}
In SLEPc, these systems may be parameterized using the -st\_ prefix, e.g.,
-st\_pc\_type~lu. If they are solved iteratively, it is possible to
use a \pk{KSP} of type \pk{KSPHPDDM} to recycle information between
successive linear systems yielded by the convergence of the Krylov--Schur
algorithm. Using the FreeFEM~\cite{hecht2012new} mini-app from \url{https://github.com/prj-/moulin2019al}
which precondition~\cref{eq:ks} using a modified augmented Lagrangian
approach~\cite{Benzi2011a}, the effect of such recycling is shown next for
finding the 5 eigenpairs closest to the shift $\sigma = 10^{-6} + 0.6\mathfrak{i}$ for a
flow past a cylinder at Reynolds 100, interested readers are again referred to
the previously cited paper~\cite{moulin2018al} for more details and for larger
runs. Results can be reproduced using the following four commands, the first two
being merely for generating the base flow $q_b$ with \pk{SNESSolve} using a
continuation method on the Reynolds number. \\[-4pt]
\begin{Verbatim}[fontsize=\footnotesize,frame=single,framerule=0.1mm,commandchars=&\[\]]
&fvtextcolor[mygreen][$] &fvtextcolor[myyellow][mpirun] -n 4 FreeFem++-mpi Nonlinear-solver.edp -Re 50 -v 0 
&fvtextcolor[mygreen][$] &fvtextcolor[myyellow][mpirun] -n 4 FreeFem++-mpi Nonlinear-solver.edp -Re 100 -v 0 
&fvtextcolor[mygreen][$] &fvtextcolor[myyellow][mpirun] -n 4 FreeFem++-mpi Eigensolver.edp -Re 100 -v 0 -st_ksp_rtol 1.0e-4         &fvtextcolor[myred][\] 
  -st_ksp_type fgmres -st_ksp_gmres_restart 200 -st_ksp_converged_reason
&fvtextcolor[mygreen][$] &fvtextcolor[myyellow][mpirun] -n 4 FreeFem++-mpi Eigensolver.edp -Re 100 -v 0 -st_ksp_rtol 1.0e-4         &fvtextcolor[myred][\] 
  -st_ksp_type hpddm -st_ksp_gmres_restart 200 -st_ksp_converged_reason              &fvtextcolor[myred][\] 
  -st_ksp_hpddm_variant flexible -st_ksp_hpddm_recycle 10 -st_ksp_hpddm_type gcrodr
\end{Verbatim}
The Krylov--Schur algorithm converges in six outer iterations. In each of these
iterations, multiple linear systems having the structure of~\cref{eq:ks} are
solved. In~\cref{fig:comparison-slepc}, numbers of iterates per inner solves
are reported. After the first inner solve, once recycling kicks in, one may
clearly see the benefits of switching from \pk{KSPFGMRES} to \pk{KSPHPDDM}.
Indeed, through the 46 inner solves, these numbers are roughly decreased by 25. In terms of runtime, \SI[round-mode          = places,round-precision=1]{1.5098e+03}{\sec} (resp.~\SI[round-mode          = places,round-precision=1]{1.0524e+03}{\sec}) is spent in \pk{EPSSolve} when using \pk{KSPFGMRES} (resp.~\pk{KSPHPDDM}). These timings, and all which follow, have been obtained on {Ir\`ene}, a system composed of \pgfmathprintnumber[assume math mode=true]{1656} nodes with two 24-core Intel Xeon Platinum 8168 clocked at \SI{2.7}{\giga\hertz}. 
\pgfplotstableread{
sys hpddm fgmres
1   77    77
2   54    83
3   53    83
4   52    81
5   52    81
6   51    78
7   51    79
8   51    80
9   51    80
10  51    77
11  51    79
12  51    79
13  51    79
14  53    77
15  52    79
16  49    78
17  51    77
18  51    79
19  52    79
20  52    79
21  51    77
22  51    78
23  51    79
24  53    80
25  54    79
26  52    79
27  49    75
28  51    78
29  51    78
30  51    77
31  50    78
32  55    81
33  56    79
34  54    80
35  53    80
36  53    78
37  50    78
38  52    79
39  53    77
40  52    77
41  54    80
42  53    80
43  55    80
44  53    79
45  52    77
46  53    80
}\loadedtableSLEPc
    \begin{figure}
\centering \begin{tikzpicture}
    \begin{axis}[xlabel={Krylov--Schur system index},ylabel={\# of
        iterations},xmin=1,xmax=46,ymin=0,ymajorgrids,
        legend columns=1,enlarge x limits=0.05,
        legend style={font=\small,at={(0.5,0.24)}, legend cell align=left,
        anchor=south west}]
\addplot[very thick,last]     table[x expr=\thisrowno{0},y expr=\thisrowno{2}, col sep=space] {\loadedtableSLEPc};\label{pgfplots:m0}
\addplot[very thick,myred,densely dashed]       table[x expr=\thisrowno{0},y expr=\thisrowno{1}, col sep=space] {\loadedtableSLEPc};\label{pgfplots:m1}
\legend{{\pk{KSPFGMRES}},{\pk{KSPHPDDM}}}
\end{axis}
\end{tikzpicture}
        \caption{\# of inner iterations for Krylov--Schur systems~\cref{eq:ks} with~\ref{pgfplots:m0} or without~\ref{pgfplots:m1} recycling.\label{fig:comparison-slepc}}
    \end{figure}
\pgfplotstableread{
N    baij-1-T    baij-1-F   sbaij-1-T   sbaij-1-F  aij-1-T     aij-1-F    baij-3-T    baij-3-F   sbaij-3-T   sbaij-3-F  aij-3-T     aij-3-F    baij-10-T   baij-10-F  sbaij-10-T  sbaij-10-F aij-10-T    aij-10-F     cusparse-1-T    cusparse-1-F    cusparse-3-T    cusparse-3-F    cusparse-10-T   cusparse-10-F    aijmkl-seq-1-T      aijmkl-seq-3-T      aijmkl-seq-10-T     baijmkl-seq-1-T      baijmkl-seq-3-T      baijmkl-seq-10-T   sbaijmkl-seq-1-T      sbaijmkl-seq-3-T      sbaijmkl-seq-10-T         aijmkl-1-T      aijmkl-3-T      aijmkl-10-T     baijmkl-1-T      baijmkl-3-T      baijmkl-10-T   sbaijmkl-1-T      sbaijmkl-3-T      sbaijmkl-10-T     
1      2.0126e-01   1.4696e+08 8.0098e-02   1.4696e+08   9.3388e-02   1.4696e+08     8.9584e-01   1.3535e+09    6.2428e-01   1.3638e+09     1.2072e+00   1.3535e+09   4.0734e+00   5.4450e+09    2.2476e+00   5.4708e+09    5.0428e+00   5.4450e+09   4.4800e-02   1.5211e+08   3.7933e-01   1.3690e+09   1.5043e+00   5.4759e+09   6.6857e-02  7.4649e-01   3.1128e+00   4.7672e-01   6.2903e-01  2.2271e+00   4.7675e-01   6.2990e-01    2.2273e+00    7.8399e-03    8.6487e-02    3.5536e-01   5.2814e-02 6.6225e-02   2.4584e-01   4.5115e-02   6.6404e-02   2.4489e-01
2      3.2635e-01   2.9392e+08 2.4204e-01   3.2482e+08   1.1522e-01   3.0422e+08     1.2835e+00   2.7071e+09    1.0697e+00   2.9234e+09     9.9480e-01   2.7380e+09   1.4504e+01   1.0890e+10    1.5265e+01   1.1694e+10    3.7799e+00   1.0952e+10   5.2164e-03   3.0422e+08   3.4056e-02   2.7380e+09   7.7659e-02   1.0952e+10   1.1703e-01  9.3872e-01   3.5976e+00   4.7905e-01   7.3497e-01  3.3331e+00   4.7886e-01   7.3415e-01    3.3297e+00    1.2450e-02    1.0124e-01    3.9873e-01   4.7633e-02 7.4107e-02   2.8513e-01   4.7733e-02   7.4693e-02   2.8507e-01
8      7.9573e-01   1.1757e+09 1.0070e+00   1.2993e+09   3.3729e-01   1.2169e+09     3.3663e+00   1.0828e+10    3.9830e+00   1.1694e+10     2.4656e+00   1.0952e+10   2.2209e+01   4.3560e+10    2.2545e+01   4.6775e+10    9.2040e+00   4.3807e+10   1.3652e-02   1.2169e+09   8.2139e-02   1.0952e+10   1.8984e-01   4.3807e+10   4.2722e-01  2.8760e+00   1.0913e+01   1.3256e+00   1.7457e+00  7.9644e+00   1.3210e+00   1.7410e+00    7.9679e+00    4.2997e-02    2.6899e-01    1.6814e+00   1.3183e-01 1.7832e-01   8.8593e-01   1.3082e-01   1.7947e-01   8.8420e-01
16     2.1378e+00   2.3513e+09 3.0837e+00   2.5986e+09   6.7295e-01   2.4337e+09     9.1522e+00   2.1656e+10    1.1762e+01   2.3387e+10     4.9326e+00   2.1904e+10   3.1530e+01   8.7120e+10    2.9334e+01   9.3549e+10    1.8462e+01   8.7615e+10   2.4907e-02   2.4337e+09   1.3832e-01   2.1904e+10   3.3609e-01   8.7615e+10   8.5514e-01  5.7493e+00   2.1980e+01   2.6391e+00   3.4887e+00  1.5947e+01   2.6455e+00   3.4893e+00    1.5928e+01    8.4255e-02    5.4284e-01    3.3861e+00   2.6162e-01 3.5777e-01   1.7955e+00   2.6238e-01   3.5880e-01   1.7923e+00
32     4.2781e+00   4.7026e+09 6.5426e+00   5.1972e+09   1.3447e+00   4.8675e+09     1.8545e+01   4.3313e+10    2.4015e+01   4.6775e+10     9.8656e+00   4.3807e+10   4.4678e+01   1.7424e+11    4.1021e+01   1.8710e+11    3.6822e+01   1.7523e+11   4.7533e-02   4.8675e+09   2.6294e-01   4.3807e+10   6.3836e-01   1.7523e+11   1.7075e+00  1.1487e+01   4.3644e+01   5.2831e+00   6.9603e+00  3.1855e+01   5.2907e+00   6.9739e+00    3.1852e+01    1.6480e-01    1.0807e+00    6.7653e+00   5.2012e-01 7.1464e-01   3.5401e+00   5.2109e-01   7.1542e-01   3.5382e+00
64     9.9341e+00   9.4053e+09 1.3822e+01   1.0394e+10   2.6927e+00   9.7350e+09     3.5490e+01   8.6626e+10    4.6966e+01   9.3549e+10     1.9724e+01   8.7615e+10   6.7668e+01   3.4848e+11  6.3229e+01   3.7420e+11  7.4386e+01   3.5046e+11   8.3249e-02   9.7350e+09   5.0956e-01   8.7615e+10   1.2334e+00   3.5046e+11   3.4264e+00  2.3052e+01   8.7358e+01   1.0573e+01   1.3960e+01  6.3695e+01   1.0581e+01   1.3961e+01    6.3949e+01    3.2990e-01    2.1734e+00    1.3526e+01   1.0453e+00 1.4281e+00   7.0763e+00   1.0497e+00   1.4326e+00   7.1714e+00
}\loadedMatProduct
    \subsubsection{Performance of primitives for block Krylov methods}
As detailed in~\cref{sec:interface-ksp}, it is currently not possible with most
PETSc preconditioners to apply them to a block of vectors. Even though there
are efforts to enable such an operation for an increasing number of \pk{PC}, in
this section, the focus is instead put on the matrix--matrix multiplication
needed in block Krylov methods or when restarting recycling Krylov methods.
In particular, the performance of three PETSc matrix types are evaluated:
\begin{itemize}
    \item \pk{MatSeqAIJ}, standard sequential sparse matrices, based on compressed sparse row format;
    \item \pk{MatSeqBAIJ}, block sparse matrices, based on block compressed sparse row format;
    \item \pk{MatSeqSBAIJ}, symmetric block sparse matrices.
\end{itemize}
Parallel matrices in PETSc heavily rely on sequential matrices. For example, \pk{MatMPIAIJ}, standard parallel sparse matrices, store on each rank two \pk{MatSeqAIJ}, one with local rows and columns, another with local rows and ``nonlocal'' columns.
To keep this study succinct, only sequential sparse matrices will be evaluated next, which can help draw conclusions for the intraprocess performance of parallel sparse matrices. The Seq substring is now dropped, since in this case, \pk{MatSeqAIJ} is equivalent to \pk{MatAIJ} and so on.

The Poisson equation on a cube is discretized with $\mathbb{P}_1$ finite elements in double precision. This yields a matrix $A$ of dimension one million. Though the numerical values of $A$ are of no interest here, the sparsity pattern is rather common and frequently encountered, e.g., when discretizing partial differential equations. Then, a random symmetric dense matrix $T$ of dimension $b$ is generated, for $b \in \{ 1,3,6 \}$, and the following matrix is evaluated $\mathbb{A} = A \otimes T$. The matrix $\mathbb{A}$ is then assembled into all three formats described above. For block matrices, a block size of $b$ is used. The performance of the \pk{MatProductNumeric} operation is evaluated for tall-and-skinny dense matrices with a varying number of columns $N \in \{1,2,8,16,32,64\}$. For $N=1$, the \pk{MatMult} operation for matrix--vector multiplication is used instead. All results are gathered in~\cref{fig:perfAX}. They can be reproduced using the following mini-app: \url{http://jolivet.perso.enseeiht.fr/MatProduct.cpp} and any input \pk{MatAIJ} stored in binary format, here using the default name binaryoutput.
\begin{Verbatim}[fontsize=\footnotesize,frame=single,framerule=0.1mm,commandchars=&\[\]]
&fvtextcolor[mygreen][$] &fvtextcolor[myyellow][mpicc] MatProduct.c -O3 -I&fvtextcolor[myyellow][${PETSC_DIR}]/&fvtextcolor[myyellow][${PETSC_ARCH}]/include                        &fvtextcolor[myred][\]
  -I&fvtextcolor[myyellow][${PETSC_DIR}]/include -L&fvtextcolor[myyellow][${PETSC_DIR}]/&fvtextcolor[myyellow][${PETSC_ARCH}]/lib -lpetsc -o MatProduct
&fvtextcolor[mygreen][$] &fvtextcolor[myyellow][mpirun] -n 1 ./MatProduct -f binaryoutput -log_view                                 &fvtextcolor[myred][\]
  -bs 1,3,6 -N 1,2,8,16,32,64 -type aij,aijmkl,baij,baijmkl,sbaij,sbaijmkl
\end{Verbatim}
\begin{figure}[htbp]
\pgfkeys{/pgf/fpu}
  \pgfmathparse{1.4696e+08/9.3388e-02/1000000000.0}\edef\storeresultI{\pgfmathresult}
  \pgfmathparse{1.3535e+09/8.9584e-01/1000000000.0}\edef\storeresultII{\pgfmathresult}
  \pgfmathparse{5.4450e+09/4.0734e+00/1000000000.0}\edef\storeresultIII{\pgfmathresult}
\pgfkeys{/pgf/fpu=false}
  % \hspace*{-1.5cm}
  \centering
    \begin{tikzpicture}
      \begin{semilogxaxis}[
        ymin=0,
        ybar=1.5pt,
        bar width=3.5pt,
        enlarge x limits=0.101,height=4cm,
        width=0.8\textwidth,
        ylabel={},
        legend columns=3, legend style={legend cell align=left,at={(0.5,1.03)},anchor=south},
        xtick = {\empty},
        xmajorgrids=true,
        extra x ticks = {1,2,4,8,16,32},
        extra x tick labels = {\empty},
        extra tick style={
            grid style={dotted,black}
        },
        xtick pos=left,name=plot1,
        xticklabels={\empty},xmajorticks=false,
        ytick = {0.5,1,1.5,2},
        ymajorgrids=true,yticklabel={
\pgfmathparse{int(10*\tick)}
\ifnum\pgfmathresult=0
\else
\pgfmathparse{100*\tick}
$\pgfmathprintnumber{\pgfmathresult}$\%
\fi
},
          cycle list={blueaccent, darkgray, mediumgray},%,last},
          clip=false,point meta=exp(x),
      ]
          \addplot+[nodes near coords style={xshift=0.5cm,yshift=-0.1cm,above}, nodes near coords*={\rotatebox{45}{\tiny\color{black}{\pgfmathfloattofixed{\pgfplotspointmeta}\pgfmathparse{int(round(\pgfmathresult))}\ifnum\pgfmathresult=1\pgfmathprintnumber\storeresultI~GFLOP/s\fi}}},
draw=black,fill, area legend,postaction={pattern=crosshatch dots}] table [x expr=\thisrow{N}>7?\thisrow{N}/2:\thisrow{N},y expr=\thisrow{aij-1-F}/\thisrow{aij-1-T}/(\storeresultI*1000000000.0)]{\loadedMatProduct};\label{pgfplots:aij1}
          \addplot+[draw=black,fill, area legend,postaction={pattern=crosshatch dots}] table [x expr=\thisrow{N}>7?\thisrow{N}/2:\thisrow{N},y expr=\thisrow{baij-1-F}/\thisrow{baij-1-T}/(\storeresultI*1000000000.0)]{\loadedMatProduct};\label{pgfplots:baij1}
          \addplot+[draw=black,fill, area legend,postaction={pattern=crosshatch dots}] table [x expr=\thisrow{N}>7?\thisrow{N}/2:\thisrow{N},y expr=\thisrow{sbaij-1-F}/\thisrow{sbaij-1-T}/(\storeresultI*1000000000.0)]{\loadedMatProduct};\label{pgfplots:sbaij1}
           \addplot+[draw=black,fill, area legend,postaction={pattern=bricks}] table [x expr=\thisrow{N}>7?\thisrow{N}/2:\thisrow{N},y expr=\thisrow{aij-1-F}/\thisrow{aijmkl-seq-1-T}/(\storeresultI*1000000000.0)]{\loadedMatProduct};\label{pgfplots:mklaijseq1}
           \addplot+[draw=black,fill, area legend,postaction={pattern=bricks}] table [x expr=\thisrow{N}>7?\thisrow{N}/2:\thisrow{N},y expr=\thisrow{baij-1-F}/\thisrow{baijmkl-seq-1-T}/(\storeresultI*1000000000.0)]{\loadedMatProduct};\label{pgfplots:mklbaijseq1}
           \addplot+[draw=black,fill, area legend,postaction={pattern=bricks}] table [x expr=\thisrow{N}>7?\thisrow{N}/2:\thisrow{N},y expr=\thisrow{sbaij-1-F}/\thisrow{sbaijmkl-seq-1-T}/(\storeresultI*1000000000.0)]{\loadedMatProduct};\label{pgfplots:mklsbaijseq1}
           \addlegendentry{\pk{MatAIJ}}
           \addlegendentry{\pk{MatBAIJ}}
           \addlegendentry{\pk{MatSBAIJ}}
           \addlegendentry{\pk{MatAIJMKL}}
           \addlegendentry{\pk{MatBAIJMKL}}
           \addlegendentry{MatSBAIJMKL}
\node[rotate=0,fill=white,inner sep=0pt,anchor=west] at (rel axis cs:1.02, 0.5) {$b=1$};
    \end{semilogxaxis}
      \begin{semilogxaxis}[
        ymin=0,
        ybar=1.5pt,
        bar width=3.5pt,
        enlarge x limits=0.101,height=4cm,
        width=0.8\textwidth,
        % legend columns=1, legend style={legend cell align=left,at={(1.02,0.5)},anchor=west},
        ylabel={Efficiency},
        xtick = {\empty},
        xmajorgrids=true,
        extra x ticks = {1,2,4,8,16,32},
        extra x tick labels = {\empty},
        extra tick style={
            grid style={dotted,black}
        },
        xtick pos=left,name=plot2,at={($(plot1.south)-(0,0.2cm)$)},anchor=north,
        xticklabels={\empty},xmajorticks=false,
        ytick = {1,2,3,4},
        ymajorgrids=true,yticklabel={
\pgfmathparse{int(10*\tick)}
\ifnum\pgfmathresult=0
\else
\pgfmathparse{100*\tick}
$\pgfmathprintnumber{\pgfmathresult}$\%
\fi
},
          cycle list={blueaccent, darkgray, mediumgray},%,last},
          clip=false,point meta=exp(x)
      ]
          \addplot+[draw=black,fill, area legend,postaction={pattern=crosshatch dots}] table [x expr=\thisrow{N}>7?\thisrow{N}/2:\thisrow{N},y expr=\thisrow{aij-3-F}/\thisrow{aij-3-T}/(\storeresultII*1000000000.0)]{\loadedMatProduct};\label{pgfplots:aij3}
          \addplot+[nodes near coords style={xshift=0.5cm,yshift=-0.1cm,above}, nodes near coords*={\rotatebox{45}{\tiny\color{black}{\pgfmathfloattofixed{\pgfplotspointmeta}\pgfmathparse{int(round(\pgfmathresult))}\ifnum\pgfmathresult=1\pgfmathprintnumber\storeresultII~GFLOP/s\fi}}},
              draw=black,fill, area legend,postaction={pattern=crosshatch dots}] table [x expr=\thisrow{N}>7?\thisrow{N}/2:\thisrow{N},y expr=\thisrow{baij-3-F}/\thisrow{baij-3-T}/(\storeresultII*1000000000.0)]{\loadedMatProduct};\label{pgfplots:baij3}
          \addplot+[draw=black,fill, area legend,postaction={pattern=crosshatch dots}] table [x expr=\thisrow{N}>7?\thisrow{N}/2:\thisrow{N},y expr=\thisrow{sbaij-3-F}/\thisrow{sbaij-3-T}/(\storeresultII*1000000000.0)]{\loadedMatProduct};\label{pgfplots:sbaij3}
           \addplot+[draw=black,fill, area legend,postaction={pattern=bricks}] table [x expr=\thisrow{N}>7?\thisrow{N}/2:\thisrow{N},y expr=\thisrow{aij-3-F}/\thisrow{aijmkl-seq-3-T}/(\storeresultII*1000000000.0)]{\loadedMatProduct};\label{pgfplots:mklaijseq3}
           \addplot+[draw=black,fill, area legend,postaction={pattern=bricks}] table [x expr=\thisrow{N}>7?\thisrow{N}/2:\thisrow{N},y expr=\thisrow{baij-3-F}/\thisrow{baijmkl-seq-3-T}/(\storeresultII*1000000000.0)]{\loadedMatProduct};\label{pgfplots:mklbaijseq3}
           \addplot+[draw=black,fill, area legend,postaction={pattern=bricks}] table [x expr=\thisrow{N}>7?\thisrow{N}/2:\thisrow{N},y expr=\thisrow{sbaij-3-F}/\thisrow{sbaijmkl-seq-3-T}/(\storeresultII*1000000000.0)]{\loadedMatProduct};\label{pgfplots:mklsbaijseq3}
           % \addlegendentry{\pk{MatAIJ}}
           % \addlegendentry{\pk{MatBAIJ}}
           % \addlegendentry{\pk{MatSBAIJ}}
           % \addlegendentry{\pk{MatAIJMKL}}
           % \addlegendentry{\pk{MatBAIJMKL}}
           % \addlegendentry{MatSBAIJMKL}
\node[rotate=0,fill=white,inner sep=0pt,anchor=west] at (rel axis cs:1.02, 0.5) {$b=3$};
    \end{semilogxaxis}
      \begin{semilogxaxis}[
        ymin=0,
        ybar=1.5pt,
        bar width=3.5pt,
        enlarge x limits=0.101,height=4cm,
        width=0.8\textwidth,
        xlabel={\# of right-hand sides ($N$)},
        ylabel={},
        xtick = {\empty},
        xmajorgrids=true,
        extra x ticks = {1,2,4,8,16,32},
        extra x tick labels = {1,2,8,16,32,64},
        extra tick style={
            grid style={dotted,black}
        },
        xtick pos=left,name=plot3,at={($(plot2.south)-(0,0.2cm)$)},anchor=north,
        xticklabels={1,2,8,16,32,64},
        ytick = {1,2,3,4},
        ymajorgrids=true,yticklabel={
\pgfmathparse{int(10*\tick)}
\ifnum\pgfmathresult=0
\else
\pgfmathparse{100*\tick}
$\pgfmathprintnumber{\pgfmathresult}$\%
\fi
},
        %xmajorgrids=true,
          cycle list={blueaccent, darkgray, mediumgray},%,last},
          clip=false, point meta = exp(x)
      ]
      \contourlength{.6pt}
\node[rotate=0,fill=white,inner sep=0pt,anchor=west] at (rel axis cs:1.02, 0.5) {$b=6$};
          \addplot+[draw=black,fill, area legend,postaction={pattern=crosshatch dots}] table [x expr=\thisrow{N}>7?\thisrow{N}/2:\thisrow{N},y expr=\thisrow{aij-10-F}/\thisrow{aij-10-T}/(\storeresultIII*1000000000.0)]{\loadedMatProduct};\label{pgfplots:aij10}
          \addplot+[nodes near coords style={xshift=0.5cm,yshift=-0.1cm,above}, nodes near coords*={\rotatebox{45}{\tiny\color{black}{\pgfmathfloattofixed{\pgfplotspointmeta}\pgfmathparse{int(round(\pgfmathresult))}\ifnum\pgfmathresult=1\contour{white}{\pgfmathprintnumber\storeresultIII}~GFLOP/s\fi}}},draw=black,fill, area legend,postaction={pattern=crosshatch dots}] table [x expr=\thisrow{N}>7?\thisrow{N}/2:\thisrow{N},y expr=\thisrow{baij-10-F}/\thisrow{baij-10-T}/(\storeresultIII*1000000000.0)]{\loadedMatProduct};\label{pgfplots:baij10}
          \addplot+[draw=black,fill, area legend,postaction={pattern=crosshatch dots}] table [x expr=\thisrow{N}>7?\thisrow{N}/2:\thisrow{N},y expr=\thisrow{sbaij-10-F}/\thisrow{sbaij-10-T}/(\storeresultIII*1000000000.0)]{\loadedMatProduct};\label{pgfplots:sbaij10}
           \addplot+[draw=black,fill, area legend,postaction={pattern=bricks}] table [x expr=\thisrow{N}>7?\thisrow{N}/2:\thisrow{N},y expr=\thisrow{aij-10-F}/\thisrow{aijmkl-seq-10-T}/(\storeresultIII*1000000000.0)]{\loadedMatProduct};\label{pgfplots:mklaijseq10}
           \addplot+[draw=black,fill, area legend,postaction={pattern=bricks}] table [x expr=\thisrow{N}>7?\thisrow{N}/2:\thisrow{N},y expr=\thisrow{baij-10-F}/\thisrow{baijmkl-seq-10-T}/(\storeresultIII*1000000000.0)]{\loadedMatProduct};\label{pgfplots:mklbaijseq10}
           \addplot+[draw=black,fill, area legend,postaction={pattern=bricks}] table [x expr=\thisrow{N}>7?\thisrow{N}/2:\thisrow{N},y expr=\thisrow{sbaij-10-F}/\thisrow{sbaijmkl-seq-10-T}/(\storeresultIII*1000000000.0)]{\loadedMatProduct};\label{pgfplots:mklsbaijseq10}
    \end{semilogxaxis}
    \end{tikzpicture}
   \caption{Performance of the matrix--matrix multiplication as implemented in PETSc for three sparse matrix types. As baseline values with $N=1$, for $b=1$, performance of \pk{MatAIJ} matrix--vector multiplication is used~\ref{pgfplots:aij1}, while for $b=\{3,6\}$, it is the performance of \pk{MatBAIJ} matrix--vector multiplication~\ref{pgfplots:baij3}.\label{fig:perfAX}}
\end{figure}
            
\begin{figure}[htbp]
\pgfkeys{/pgf/fpu}
  \pgfmathparse{1.4696e+08/7.8399e-03/1000000000.0}\edef\storeresultI{\pgfmathresult}
  \pgfmathparse{1.3535e+09/6.6225e-02/1000000000.0}\edef\storeresultII{\pgfmathresult}
  \pgfmathparse{5.4450e+09/2.4584e-01/1000000000.0}\edef\storeresultIII{\pgfmathresult}
\pgfkeys{/pgf/fpu=false}
  \centering
    \begin{tikzpicture}
      \begin{semilogxaxis}[
        ymin=0,
        ybar=1.5pt,
        bar width=3.5pt,
        enlarge x limits=0.101,height=4cm,
        width=0.8\textwidth,
        ylabel={},
        legend columns=4, legend style={legend cell align=left,at={(0.5,1.03)},anchor=south},
        xtick = {\empty},
        xmajorgrids=true,
        extra x ticks = {1,2,4,8,16,32},
        extra x tick labels = {\empty},
        extra tick style={
            grid style={dotted,black}
        },
        xtick pos=left,name=plot4,
        xticklabels={\empty},xmajorticks=false,
        ytick = {1,3.5,6},
        ymajorgrids=true,yticklabel={
\pgfmathparse{int(10*\tick)}
\ifnum\pgfmathresult=0
\else
\pgfmathparse{100*\tick}
$\pgfmathprintnumber{\pgfmathresult}$\%
\fi
},
        %xmajorgrids=true,
          cycle list={blueaccent, darkgray, mediumgray,last},
          point meta=exp(x),clip=false
      ]
% \node[font=\tiny,rotate=45,fill=white,inner sep=0pt,anchor=west] at (axis cs:0.725, 1.2) {\pgfmathprintnumber\storeresultI~GFLOP/s};
% \node[font=\tiny,rotate=45,fill=white,inner sep=0pt,anchor=west] at (axis cs:0.95, 1.2) {\pgfmathprintnumber\storeresultII~GFLOP/s};
% \node[font=\tiny,rotate=45,fill=white,inner sep=0pt,anchor=west] at (axis cs:1.2, 1.2) {\pgfmathprintnumber\storeresultIII~GFLOP/s};
          \addplot+[nodes near coords style={xshift=0.55cm,yshift=-0.1cm,above}, nodes near coords*={\rotatebox{45}{\tiny\color{black}{\pgfmathfloattofixed{\pgfplotspointmeta}\pgfmathparse{int(round(\pgfmathresult))}\ifnum\pgfmathresult=1\pgfmathprintnumber\storeresultI~GFLOP/s\fi}}},draw=black,fill, area legend] table [x expr=\thisrow{N}>7?\thisrow{N}/2:\thisrow{N},y expr=\thisrow{aij-1-F}/\thisrow{aijmkl-1-T}/(\storeresultI*1000000000.0)]{\loadedMatProduct};\label{pgfplots:aijmkl1}
          \addplot+[draw=black,fill, area legend] table [x expr=\thisrow{N}>7?\thisrow{N}/2:\thisrow{N},y expr=\thisrow{baij-1-F}/\thisrow{baijmkl-1-T}/(\storeresultI*1000000000.0)]{\loadedMatProduct};\label{pgfplots:baijmkl1}
          \addplot+[draw=black,fill, area legend] table [x expr=\thisrow{N}>7?\thisrow{N}/2:\thisrow{N},y expr=\thisrow{sbaij-1-F}/\thisrow{sbaijmkl-1-T}/(\storeresultI*1000000000.0)]{\loadedMatProduct};\label{pgfplots:sbaijmkl1}
           \addplot+[draw=black,fill, area legend] table [x expr=\thisrow{N}>7?\thisrow{N}/2:\thisrow{N},y expr=\thisrow{aij-1-F}/\thisrow{cusparse-1-T}/(\storeresultI*1000000000.0)]{\loadedMatProduct};\label{pgfplots:cusparse1}
           \addlegendentry{\pk{MatAIJMKL}}
           \addlegendentry{\pk{MatBAIJMKL}}
           \addlegendentry{MatSBAIJMKL}
           \addlegendentry{\pk{MatAIJcuSPARSE}}
\node[rotate=0,fill=white,inner sep=0pt,anchor=west] at (rel axis cs:1.02, 0.5) {$b=1$};
    \end{semilogxaxis}
      \begin{semilogxaxis}[
        ymin=0,
        ybar=1.5pt,
        bar width=3.5pt,
        enlarge x limits=0.101,height=4cm,
        width=0.8\textwidth,
        % legend columns=1, legend style={legend cell align=left,at={(1.02,0.5)},anchor=west},
        ylabel={Efficiency},
        xtick = {\empty},
        xmajorgrids=true,
        extra x ticks = {1,2,4,8,16,32},
        extra x tick labels = {\empty},
        extra tick style={
            grid style={dotted,black}
        },
        xtick pos=left,name=plot5,at={($(plot4.south)-(0,0.2cm)$)},anchor=north,
        xticklabels={\empty},xmajorticks=false,
        ytick = {1,4.5,8},
        ymajorgrids=true,yticklabel={
\pgfmathparse{int(10*\tick)}
\ifnum\pgfmathresult=0
\else
\pgfmathparse{100*\tick}
$\pgfmathprintnumber{\pgfmathresult}$\%
\fi
},
        %xmajorgrids=true,
          cycle list={blueaccent, darkgray, mediumgray,last},
          point meta=exp(x),clip=false
      ]
% \node[font=\tiny,rotate=45,fill=white,inner sep=0pt,anchor=west] at (axis cs:0.725, 1.2) {\pgfmathprintnumber\storeresultI~GFLOP/s};
% \node[font=\tiny,rotate=45,fill=white,inner sep=0pt,anchor=west] at (axis cs:0.95, 1.2) {\pgfmathprintnumber\storeresultII~GFLOP/s};
% \node[font=\tiny,rotate=45,fill=white,inner sep=0pt,anchor=west] at (axis cs:1.2, 1.2) {\pgfmathprintnumber\storeresultIII~GFLOP/s};
          \addplot+[draw=black,fill, area legend] table [x expr=\thisrow{N}>7?\thisrow{N}/2:\thisrow{N},y expr=\thisrow{aij-3-F}/\thisrow{aijmkl-3-T}/(\storeresultII*1000000000.0)]{\loadedMatProduct};\label{pgfplots:aijmkl3}
          \addplot+[nodes near coords style={xshift=0.55cm,yshift=-0.1cm,above}, nodes near coords*={\rotatebox{45}{\tiny\color{black}{\pgfmathfloattofixed{\pgfplotspointmeta}\pgfmathparse{int(round(\pgfmathresult))}\ifnum\pgfmathresult=1\pgfmathprintnumber\storeresultII~GFLOP/s\fi}}},draw=black,fill, area legend] table [x expr=\thisrow{N}>7?\thisrow{N}/2:\thisrow{N},y expr=\thisrow{baij-3-F}/\thisrow{baijmkl-3-T}/(\storeresultII*1000000000.0)]{\loadedMatProduct};\label{pgfplots:baijmkl3}
          \addplot+[draw=black,fill, area legend] table [x expr=\thisrow{N}>7?\thisrow{N}/2:\thisrow{N},y expr=\thisrow{sbaij-3-F}/\thisrow{sbaijmkl-3-T}/(\storeresultII*1000000000.0)]{\loadedMatProduct};\label{pgfplots:sbaijmkl3}
           \addplot+[draw=black,fill, area legend] table [x expr=\thisrow{N}>7?\thisrow{N}/2:\thisrow{N},y expr=\thisrow{aij-3-F}/\thisrow{cusparse-3-T}/(\storeresultII*1000000000.0)]{\loadedMatProduct};\label{pgfplots:cusparse3}
           % \addlegendentry{\pk{MatAIJ}}
           % \addlegendentry{\pk{MatBAIJ}}
           % \addlegendentry{\pk{MatSBAIJ}}
           % \addlegendentry{\pk{MatAIJMKL}}
           % \addlegendentry{\pk{MatBAIJMKL}}
           % \addlegendentry{MatSBAIJMKL}
\node[rotate=0,fill=white,inner sep=0pt,anchor=west] at (rel axis cs:1.02, 0.5) {$b=3$};
    \end{semilogxaxis}
      \begin{semilogxaxis}[
        ymin=0,ymax=5.5,
        ybar=1.5pt,
        bar width=3.5pt,
        enlarge x limits=0.101,height=4cm,
        width=0.8\textwidth,
        xlabel={\# of right-hand sides ($N$)},
        ylabel={},
        xtick = {\empty},
        xmajorgrids=true,
        extra x ticks = {1,2,4,8,16,32},
        extra x tick labels = {1,2,8,16,32,64},
        extra tick style={
            grid style={dotted,black}
        },
        xtick pos=left,name=plot3,at={($(plot2.south)-(0,0.2cm)$)},anchor=north,
        xticklabels={1,2,8,16,32,64},
        ytick = {1,2,ln(6.5-5)+3,ln(10-5)+3},%,ln(30-5)+3},
        ylabel style={yshift=6pt},
        ymajorgrids=true,yticklabel={
\pgfmathparse{int(10*\tick)}
\ifnum\pgfmathresult=0
\else
\ifnum\pgfmathresult>20
\pgfmathparse{ceil((exp(\tick-3)+5))*100}
$\pgfmathprintnumber{\pgfmathresult}$\%
\else
\pgfmathparse{100*\tick}
$\pgfmathprintnumber{\pgfmathresult}$\%
\fi
\fi
},
        %xmajorgrids=true,
          cycle list={blueaccent, darkgray, mediumgray,last},
          point meta=exp(x),clip=false,
      ]
\node[rotate=0,fill=white,inner sep=0pt,anchor=west] at (rel axis cs:1.02, 0.5) {$b=6$};
% \node[font=\tiny,rotate=45,fill=white,inner sep=0pt,anchor=west] at (axis cs:0.725, 1.2) {\pgfmathprintnumber\storeresultI~GFLOP/s};
% \node[font=\tiny,rotate=45,fill=white,inner sep=0pt,anchor=west] at (axis cs:0.95, 1.2) {\pgfmathprintnumber\storeresultII~GFLOP/s};
% \node[font=\tiny,rotate=45,fill=white,inner sep=0pt,anchor=west] at (axis cs:1.2, 1.2) {\pgfmathprintnumber\storeresultIII~GFLOP/s};
          \addplot+[draw=black,fill, area legend] table [x expr=\thisrow{N}>7?\thisrow{N}/2:\thisrow{N},y expr=\thisrow{aij-10-F}/\thisrow{aijmkl-10-T}/(\storeresultIII*1000000000.0)]{\loadedMatProduct};\label{pgfplots:aijmkl10}
          \addplot+[nodes near coords style={xshift=0.55cm,yshift=-0.1cm,above}, nodes near coords*={\rotatebox{45}{\tiny\color{black}{\pgfmathfloattofixed{\pgfplotspointmeta}\pgfmathparse{int(round(\pgfmathresult))}\ifnum\pgfmathresult=1\pgfmathprintnumber\storeresultIII~GFLOP/s\fi}}},draw=black,fill, area legend] table [x expr=\thisrow{N}>7?\thisrow{N}/2:\thisrow{N},y expr=\thisrow{baij-10-F}/\thisrow{baijmkl-10-T}/(\storeresultIII*1000000000.0)]{\loadedMatProduct};\label{pgfplots:baijmkl10}
          \addplot+[draw=black,fill, area legend] table [x expr=\thisrow{N}>7?\thisrow{N}/2:\thisrow{N},y expr=\thisrow{sbaij-10-F}/\thisrow{sbaijmkl-10-T}/(\storeresultIII*1000000000.0)]{\loadedMatProduct};\label{pgfplots:sbaijmkl10}
          %\addplot+[draw=black,fill, area legend] table [x expr=\thisrow{N}>7?\thisrow{N}/2:\thisrow{N},y expr=\thisrow{aij-10-F}/\thisrow{cusparse-10-T}/(\storeresultIII*1000000000.0)]{\loadedMatProduct};\label{pgfplots:cusparse10}
          \addplot+[draw=black,fill, area legend] table [x expr=\thisrow{N}>7?\thisrow{N}/2:\thisrow{N},y expr=\thisrow{N}>1.5?ln(\thisrow{aij-10-F}/\thisrow{cusparse-10-T}/(\storeresultIII*1000000000.0)-5.0)+3:\thisrow{aij-10-F}/\thisrow{cusparse-10-T}/(\storeresultIII*1000000000.0)]{\loadedMatProduct};\label{pgfplots:cusparse10}
          \begin{pgfonlayer}{foreground}
\begin{scope}[decoration={snake,amplitude=1pt,pre length=0,post length=0}]
          \path[shape=coordinate]
(rel axis cs:-0.001,0.49) coordinate(b1) (rel axis cs: 1.001,0.49) coordinate(b2)
(rel axis cs: 1.001,0.51) coordinate(b3) (rel axis cs:-0.001,0.51) coordinate(b4);
  \path[fill=white] (b2) -- (b3) 
                decorate {-- (b4)}  
                -- (b1) 
                decorate {-- (b2)};
  \draw         (b3) 
                decorate {-- (b4)};  
  \draw         (b1) 
                decorate {-- (b2)};
\end{scope}
          \end{pgfonlayer}{foreground}
    \end{semilogxaxis}
    \end{tikzpicture}
   \caption{Performance of the matrix--matrix multiplication as implemented in PETSc for three sparse matrix types. As baseline values with $N=1$, for $b=1$, performance of \pk{MatAIJMKL} matrix--vector multiplication is used~\ref{pgfplots:aijmkl1}, while for $b=\{3,6\}$, it is the performance of \pk{MatBAIJMKL} matrix--vector multiplication~\ref{pgfplots:baijmkl3}.\label{fig:perfAXpar}}
\end{figure}
\section{PCHPDDM}\label{sec:PCHPDDM}
  \subsection{Related work}
Domain decomposition methods are, alongside multigrid methods, one of the
dominant paradigms for defining efficient and robust preconditioners in
modern large-scale applications dealing with partial differential equations.
In PETSc, preconditioners inherit from the \pk{PC} base class.
Historically, the block Jacobi and overlapping Schwarz methods were implemented
as \pk{PCBJACOBI} and \pk{PCASM} respectively. ASM stands for additive Schwarz
method, as originally introduced by Schwarz~\cite{schwarz1870alternating}, but
the PETSc implementation is more versatile in the sense that one can easily
switch between this method and other variants such as the restricted additive
Schwarz method~\cite{cai1999restricted}. More details about overlapping domain decomposition
preconditioners will be given in~\cref{sec:interface-pc}, but first the other
related preconditioners will be quickly presented.

In PETSc, there is also an implementation of two-level overlapping Schwarz
methods using a generalized Dryja, Smith, Widlund
preconditioner~\cite{dohrmann2008domain}. Its type is \pk{PCEXOTIC}. When it comes to
nonoverlapping preconditioners, \pk{PCNN} implements a basic balancing
Neumann--Neumann method~\cite{mandel1993balancing}. Much more advanced
techniques are available in \pk{PCBDDC}~\cite{zampini2016pcbddc}: adaptive selection
of primal constraints~\cite{pechstein2017unified}, change of basis for problems in
$H(\text{curl})$~\cite{dohrmann2016bddc}$\ldots$ For multigrid,
\pk{PCMG} exposes a machinery for geometric or algebraic preconditioners. In
particular, it is used for the implementation of the smoothed-aggregation
\pk{PCGAMG}~\cite{adams2004ultra}. There are also interfaces to other well-known
multigrid packages such as \pk{PCHYPRE} or \pk{PCML}~\cite{gee2006ml} from Trilinos. Of
course, there are many other available domain decomposition preconditioners,
either accessible through high-level libraries, e.g., FROSch~\cite{Heinlein:2016:PIT} in
Trilinos, or as stand-alone packages, e.g., BDDCML~\cite{vsistek2013parallel} or
FEMPAR-BDDC~\cite{badia2014highly}.

HPDDM implements the same one-level overlapping Schwarz methods as PETSc, albeit
emphasis is also put on optimized Schwarz methods~\cite{gander2006optimized}.
The main theoretical framework for defining robust two-level methods is called GenEO,
for generalized eigenvalue problem on the overlap~\cite{spillane2011robust,spillane2013abstract}.
GenEO was also implemented in other libraries~\cite{butler2020high}, but to the best of our knowledge,
the interface between HPDDM and PETSc proposes the first high-level access to
GenEO so that it can, for example, be used as a block-decomposed
solver~\cite{brown2012composable}.

  \subsection{Interfacing HPDDM overlapping Schwarz methods in PETSc\label{sec:interface-pc}}\
\pk{PCHPDDM}, the interface between HPDDM overlapping Schwarz methods and PETSc, is automatically
registered since PETSc version 3.12 when one configures PETSc with the extra
flag -{}-download-hpddm -{}-download-slepc. It is then
possible to select the corresponding \pk{PC}
using the command line option -pc\_type hpddm, or the routine \pk{PCSetType}(pc,
\pk{PCHPDDM}) and its Fortran equivalent. In this section, it is assumed that one
wants to solve 
\begin{equation}
    Ax = b,
\end{equation}
for a given coefficient matrix $A$ and right-hand side $b$ of dimension $n$
assembled on $N$ processes, using a ksp($A$, $P_A$), i.e., the matrix used to
define the preconditioner is $A$ itself. Without anything more from the user,
\pk{PCHPDDM} is strictly equivalent to \pk{PCASM}. That is, if $A$ is distributed among
$N$ processes the preconditioner $M^{-1}$ is:
\begin{equation}\label{eq:one}
    M^{-1} = \sum_{i = 1}^N \tilde{R}_i^T (R_i A R_i^T)^{-1} R_i,
\end{equation}
where $\{R_i\}_{i=1}^N$ are restriction operators that act on a global vector and
return a vector local to each process, possibly with some overlap.
$\{\tilde{R}_i\}_{i=1}^N$ are the same operators except that coefficients on the
overlap are set to 0. The action of the inverse of all $\{R_i A R_i^T\}_{i=1}^N$
can be parameterized through a local \pk{KSP}. If it is assumed that $A$ is the
discretization of a linear operator $\mathcal{L}$, e.g., $\mathcal{L}=-\Delta$, on a domain $\Omega$, it is
extremely common to exhibit parallelism by distributing $\Omega$ on $N$
processes, possibly with some overlap, and to construct the $\{R_i\}_{i=1}^N$ so
that they map global unknowns to unknowns local to each process. In addition to
the local operators $\{R_i A R_i^T\}_{i=1}^N$, GenEO needs users to supply
the local unassembled matrices $\{\mathring{A}_i\}_{i=1}^N$, also known as
Neumann matrices. These are yielded by the
discretization of $\mathcal{L}$ on each subdomain, with added natural boundary
conditions on the subdomain boundaries that are not boundaries of the initial
domain $\Omega$. In the case of the 1D Poisson
equation with homogeneous Dirichlet boundary conditions discretized using $\mathbb{P}_1$ finite elements, an explicit representation of these various operators is given~\cref{fig:poisson} for a
two-domain decomposition.
\begin{figure}[t]
    \begin{minipage}[t]{0.495\textwidth}
\begin{center}
    Domain and global unknown numbering \\[0.3cm]
\begin{tikzpicture}[darkstyle/.style={circle,draw,fill=gray!40,line
    width=0.2mm,font=\scriptsize,minimum size=12pt,inner sep=0pt}]
    \foreach \x in {0,...,\dof}
        \pgfmathsetmacro\result{int(\x + 1)}
        \node[darkstyle] (\x) at (1.25*\x,0) {\result};
    \pgfmathsetmacro\result{\dof - 1}
    \foreach \x [count=\xi] in {0,...,\result}
        \draw[double=black, double distance=0.01mm] (\x) -- (\xi);
\end{tikzpicture}
\\[0.5cm]
    Two-way element-based partitioning \\[0.1cm]
\begin{tikzpicture}[darkstyle/.style={circle,draw,fill=gray!40,line
    width=0.2mm,minimum size=12pt, inner sep=0pt}]
    \foreach \x in {0,...,\dof}
        \node[darkstyle] (\x) at (1.25*\x,0) {};
    \foreach \sub/\c in {0/red!50,1/blue!50} {
        \pgfmathsetmacro\start{int(\sub * round(\dof / 2))}
        \ifnum\sub=1
            \pgfmathsetmacro\finish{int(\dof - 1)}
        \else
            \pgfmathsetmacro\finish{int(min((\sub + 1) * round(\dof / 2), \dof) - 1)}
        \fi
        \foreach \x [count=\xi from \start+1] in {\start,...,\finish}
            \draw[double=\c, double distance=0.2mm] (\x) -- (\xi);
        \pgfmathsetmacro\finish{int(\finish+1)}
        \pgfmathsetmacro\sub{int(\sub+1)}
        \draw[{Tee Barb[left]}-{Tee Barb[right]}, shorten <= -0.1cm, shorten >= -0.1cm] ([yshift=0.5cm]\start.east) -- node[fill=white,inner sep=1mm,midway] {$\Omega_\sub$} ([yshift=0.5cm]\finish.west);
    }
\end{tikzpicture}
\end{center}
\begin{equation*}
    {\scriptsize A =
    \begin{bmatrix}1&0&0&0&0\\0&2&-1&0&0\\0&-1&2&-1&0\\0&0&-1&2&0\\0&0&0&0&1\end{bmatrix}}
\end{equation*}
    \end{minipage}
    \begin{minipage}[t]{0.495\textwidth}
\begin{center}
    Local numbering with overlap \\[0.1cm]
\begin{tikzpicture}[darkstyle/.style={circle,font=\scriptsize,draw,fill=gray!40,line
    width=0.2mm,minimum size=12pt,inner sep=0}]
    \foreach \conf/\c in {0/red!50,1/blue!50} {
        \pgfmathsetmacro\start{int(max(min(\conf * round(\dof / 2) - \overlap, \dof - 1), 0))}
        \ifnum\conf=1
            \pgfmathsetmacro\finish{int(\dof - 1)}
        \else
            \pgfmathsetmacro\finish{int(max(0, min((\conf + 1) * round(\dof / 2) + \overlap, \dof) - 1))}
        \fi
        \foreach \x in {0,...,\dof} {
            \ifnum\conf=1
                \pgfmathparse{ifthenelse(\x < \start || \x - 1 > \finish, "",
                int(\finish - \x + 2))} \let\nn\pgfmathresult
            \else
                \pgfmathparse{ifthenelse(\x < \start || \x - 1 > \finish, "",
                int(\x - \start + 1))} \let\nn\pgfmathresult
            \fi
            \pgfmathparse{ifthenelse(\x < \start || \x - 1 > \finish, "0.25", "1")} \let\oo\pgfmathresult
            \node[darkstyle, opacity = \oo] (\x\conf) at (1.25*\x,-1.25cm*\conf)
            {\nn};
        }
        \pgfmathsetmacro\loop{int(\dof - 1)}
        \foreach \x [count=\xi from 1] in {0,...,\loop} {
            \pgfmathparse{ifthenelse(\x < \start || \x > \finish, "black!25", "\c")} \let\cc\pgfmathresult
            \pgfmathparse{ifthenelse(\x < \start || \x > \finish, "0.25", "1")} \let\oo\pgfmathresult
            \ifnum\conf=1
                \pgfmathparse{ifthenelse(\x + 1 < \start || \x > \finish, "", int(\finish - \x))} \let\nn\pgfmathresult
            \else
                \pgfmathparse{ifthenelse(\x < \start || \x - 1 > \finish, "", int(\x - \start))} \let\nn\pgfmathresult
            \fi
            \pgfmathparse{ifthenelse(\conf == 1, "0.875cm", "-0.875cm"} \let\ss\pgfmathresult
            \draw[double=\cc, opacity=\oo, double distance=0.2mm] (\x\conf) -- node[xshift=\ss, yshift=0.2cm, inner sep=1mm,midway, opacity=1] {} (\xi\conf);
        }
    }
\end{tikzpicture}
\begin{equation*}
    {\scriptsize R_2 =
    \begin{bmatrix}0&0&0&0&1\\0&0&0&1&0\\0&0&1&0&0\\0&1&0&0&0\end{bmatrix} \;
        \tilde{R}_2 =
    \begin{bmatrix}0&0&0&0&1\\0&0&0&1&0\\0&0&1&0&0\\0&0&0&0&0\end{bmatrix}}
\end{equation*}
\begin{equation*}
    \hspace*{-0.5cm}  {\scriptsize \mathring{A}_i =
    \begin{bmatrix}1&0&0&0\\0&2&-1&0\\0&-1&2&-1\\0&0&-1&1\end{bmatrix} \;
      R_2 A R_2^T =
    \begin{bmatrix}1&0&0&0\\0&2&-1&0\\0&-1&2&-1\\0&0&-1&2\end{bmatrix}}
\end{equation*}
\end{center}
    \end{minipage}
\caption{Notations and basics of overlapping Schwarz methods.\label{fig:poisson}}
\end{figure}
Then, it is possible to enrich the original one-level preconditioner
from~\cref{eq:one} using a spectral coarse grid built in the following way:
\begin{enumerate}
    \item have SLEPc solve the local generalized eigenvalue problem concurrently:
        \begin{equation}\label{eq:gevp}
            \mathring{A}_i \Lambda_i = \lambda_i \tilde{R}_i R_i^T (R_i A R_i^T) \tilde{R}_i R_i^T \Lambda_i ;
        \end{equation}
    \item retrieve the $\nu_i$ smallest eigenpairs $\{\Lambda_{i_j}, \lambda_{i_j}\}_{j=1}^{\nu_i}$ and assemble a local deflation dense matrix $W_i = \begin{bmatrix} \tilde{R}_i R_i^T \Lambda_{i_1} & \cdots & \tilde{R}_i R_i^T \Lambda_{i_{\nu_j}} \end{bmatrix}$;
    \item define a global deflation matrix $P = \begin{bmatrix} R_1^T W_1 & \cdots & R_N^T W_N\end{bmatrix}$ and a new two-level preconditioner using the Galerkin product of $A$ and $P$:
        \begin{equation}\label{eq:two}
            M'^{-1}_{\text{additive}} = P \left(P^T A P\right)^{-1} P^T + \sum_{i = 1}^N \tilde{R}_i^T (R_i A R_i^T)^{-1} R_i.
        \end{equation}
\end{enumerate}
Note that $\{\tilde{R}_i R_i^T\}_{i=1}^N$ define a partition of unity, i.e.,
\begin{equation*}
    \sum_{i = 1}^N R_i^T \tilde{R}_i R_i^T R_i = I.
\end{equation*}
When used without PETSc and thus SLEPc, step \#1 is solved in HPDDM with
ARPACK~\cite{lehoucq1998arpack}. Step \#3 is common to most domain
decomposition or multigrid preconditioners. However, in HPDDM, the special
structure of the deflation matrix~$P$ which is sparse with dense blocks is
exploited. Interested readers are referred to the paper describing the technical
details of the implementation~\cite{jolivet2013scalable}. One important point to
notice is that the size of the coarse operator $P^TAP$ is much lower
than $n$, the size of $A$. It is thus common to have the coarse operator distributed on
fewer processes than for $A$, see for example the concept of telescoping later
introduced in PETSc~\cite{may2016extreme}. In HPDDM, the coarse operator is
computed and redistributed simultaneously. This could be technical to implement
in PETSc as the result of the routine \pk{MatPtAP} is distributed on the same communicator as the one of the input matrix $A$.

Of course, users do not have to implement the three previous steps themselves.
The auxiliary operators $\{\mathring{A}_i\}_{i=1}^N$ have simply to be provided
as well as the correspondence from local to global numbering,
see~\cref{fig:poisson}, using the routine \pk{PCHPDDMSetAuxiliaryMat}.  If there
is a \pk{DM} attached to the \pk{PC} of type \pk{DMPLEX}~\cite{knepley2009mesh}
or \pk{DMP4EST}~\cite{BursteddeWilcoxGhattas11}, then these auxiliary
overlapping operators and the correspondence are automatically assembled by
PETSc for users. Other \pk{DMType} may be supported by implementing the proper
\pk{DMCreateNeumannOverlap} callback. With the local matrix pencils available,
new options will be automatically registered such as
-pc\_hpddm\_levels\_1\_eps\_nev or -pc\_hpddm\_levels\_1\_eps\_threshold, which
may be used to set how many eigenpairs SLEPc should compute in step \#2.  The
default -eps\_nev value is 20, i.e., the dimension of the coarse space is
$20\times N$ under the assumption that all concurrent SLEPc solves converge.
Another important option is -pc\_hpddm\_coarse\_p, which sets the size of the
communicator on which the coarse operator should be distributed. The default
value is 1, i.e., the matrix $P^TAP$ is centralized on a single process.
The action of both inverses from~\cref{eq:two} can be parameterized through the
-pc\_hpddm\_coarse\_ksp\_type and -pc\_hpddm\_levels\_1\_ksp\_type options
respectively.
    \subsubsection{Coarse corrections and eigenvalue problems}
Some other parameters may be adjusted to define how the one-level preconditioner
and the coarse grid operator interact with each other. In fact, the additive
correction from~\cref{eq:two} is the simplest to write down, but most often not
very efficient numerically~\cite{tang2009comparison}, and thus not used as
default in \pk{PCHPDDM}. Users can choose between
$M'^{-1}_{\text{additive}}$ and:
\begin{align*}
    M'^{-1}_{\text{deflated}} &= Q + M^{-1}\left(I - AQ\right)\quad
    \text{(default)}\\
    M'^{-1}_{\text{balanced}} &= Q + \left(I - AQ\right)M^{-1}\left(I - AQ\right),
\end{align*}
with $Q = P \left(P^T A P\right)^{-1} P^T$. These preconditioners may be selected using the additional
options -pc\_hpddm\_coarse\_correction (deflated$|$additive$|$balanced) or the
routine \pk{PCHPDDMSetCoarseCorrectionType}(pc, type) and its Fortran equivalent.

The local generalized eigenvalue problems~\cref{eq:gevp} solved by SLEPc can also be adjusted.
For example, in addition to the local unassembled Neumann matrices
$\{\mathring{A}_i\}_{i=1}^N$, if one has access to local matrices with optimized
boundary conditions, e.g., Robin transmission conditions, $\{B_i\}_{i=1}^N$, it
is possible to pass these using the routine \pk{PCHPDDMSetRHSMat}(pc, $B_i$).
The following generalized eigenvalue problems, called GenEO-2 in the
literature~\cite{haferssas2017soras}, are then solved in each subdomain:
\begin{equation*}
    \mathring{A}_i \Lambda_i = \lambda_i B_i \Lambda_i.
\end{equation*}
    \subsubsection{Multilevel extension}\label{sec:ml}
In the previous sections, it was shown how to supply the required information to
\pk{PCHPDDM} in order to build robust two-level overlapping domain decomposition preconditioners.
A multilevel extension of the GenEO framework is also
possible~\cite{aldaas2019multi}. It can be used without any additional
information provided by users. Since the dimension of the
coarse operator of the second level grows with the number of subdomains of the
fine level, switching to a multilevel scheme may alleviate the increasing cost
of solving coarse systems. \pk{PCHPDDM} follows the same numbering of
\pk{PCBDDC}\footnote{which itself follows the opposite numbering of \pk{PCMG}}: the
finest level is always numbered 1, the level index increases, up until the
coarsest level whose options are always prefixed by coarse\_. In order to
register an additional level $l' = l+1$, the following conditions must be met at
level $l$:
\begin{itemize}
    \item there must be more than one subdomain;
    \item at least one local eigenvector must be computed.
\end{itemize}
Here are some examples with $N = 16$ subdomains on the finest level:
\begin{itemize}
    \item -pc\_hpddm\_levels\_1\_eps\_threshold 0.4 -pc\_hpddm\_coarse\_p 8 would
        define a two-level method with the coarse operator being distributed
        among 8 processes, assuming that there is globally enough $\lambda_i$ in~\cref{eq:gevp} lower than 0.4;
    \item -pc\_hpddm\_levels\_1\_eps\_nev 10 -pc\_hpddm\_levels\_2\_p 8
        -pc\_hpddm\_levels\_2\_eps\_nev 10 -pc\_hpddm\_coarse\_p 2 would define
        a three-level method with the second level (resp.~ coarse operator)
        being distributed among 8 (resp.~2) processes;
    \item -pc\_hpddm\_levels\_1\_eps\_nev 10 -pc\_hpddm\_levels\_2\_p 8 is not
        valid because there are only two levels, so the options levels\_2\_ are
        not registered and coarse\_ has to be used instead;
    \item -pc\_hpddm\_levels\_1\_eps\_nev 10 -pc\_hpddm\_levels\_2\_eps\_nev 10 -pc\_hpddm\_coarse\_p 2 
        is not valid because there is no option levels\_2\_p, so the default
        value of 1 is used for the second level, and it is thus not possible to
        define a third level with a single second-level subdomain.
\end{itemize}
    \subsubsection{MatIS}
Because of the intrinsic nature of balancing domain
decomposition methods, \pk{PCNN} and \pk{PCBDDC} must be supplied local unassembled
matrices. These matrices are equivalent to the $\{\mathring{A}_i\}_{i=1}^N$
defined \cref{sec:interface-pc}, assuming the domain decomposition is without
overlap, see for example $\Omega_1$ and $\Omega_2$ from~\cref{fig:poisson} in a
finite element context. Using a local to global mapping, it is possible to
reconstruct the local assembled matrices, equivalent to the $\{R_i A
R_i^T\}_{i=1}^N$ from~\cref{sec:interface-pc}. All tools needed by the GenEO
framework are thus readily available. Note however that one-level overlapping
Schwarz methods are known for converging slowly when there is in fact no
overlap. It is common in the field of BDD to see the coarse problem as an
unassembled operator as well. In fact, this is mandatory to define efficient
multilevel variants, as implemented in \pk{PCBDDC}. In~\cref{sec:bddc}, it will be
shown how a coarse problem obtained in \pk{PCBDDC} may be solved using
\pk{PCHPDDM}.
  \subsection{Applications and numerical results}
    \subsubsection{System of elasticity}\label{sec:linear-elasticity}
Considering only small displacements, this linear problem with highly heterogeneous elastic moduli is solved in 3D.
Its strong formulation is given by:
\begin{align}
	\label{eq:elasticity}
	\begin{split}
		\text{div }\sigma (u)  + f	&= 0	\quad \text{in } \Omega,\\
		u														&= 0	\quad \text{on } \Gamma_D, \\
		\sigma (u) \cdot n					&= 0	\quad \text{on } \Gamma_N.
	\end{split}
\end{align}
The physical domain $\Omega$ is a beam of dimensions $[0,6] \times [0,1] \times [0,1]$.
The Cauchy stress tensor $\sigma$ is given by Hooke’s law: it can be expressed in terms of Young’s modulus $E$ and Poisson’s ratio $\nu$.
\[
	\sigma_{ij}(u) = 
	\begin{cases}
		2\mu \varepsilon_{ij}(u)													\quad	& i \ne j, \\
		2\mu \varepsilon_{ii}(u) + \lambda \text{div}(u)	\quad	& i = j,
	\end{cases}
\]
where
\[
	\varepsilon_{ij}(u) = \frac{1}{2} \left(\frac{\partial u_i}{\partial x_i} + \frac{\partial u_j}{\partial x_j} \right), \mu = \frac{E}{2(1+\nu)}, \text{ and }\lambda = \frac{E\nu}{1-2\nu}.
\]
$\Gamma_D$ is the subset of the boundary of $\Omega$ corresponding to $x = 0$.
$\Gamma_N$ is defined as the complementary of $\Gamma_D$ with respect to the boundary of $\Omega$.
\Cref{eq:elasticity} is discretized using 
$(\mathbb{P}_1, \mathbb{P}_1, \mathbb{P}_1)$ finite elements.
The physical domain $\Omega$ is decomposed in $13{,}824$ subdomains using the automatic graph partitioner ParMETIS~\cite{karypis1998fast}.
The number of unknowns is $\pgfmathprintnumber{593} \times 10^6$, with approximately $\pgfmathprintnumber{45}$ nonzero coefficients per row.
There are heterogeneities due to jumps in $E$ and $\nu$. The following discontinuous piecewise constant values are considered: $(E_1, \nu_1) = (2\times 10^{11}, 0.25)$, $(E_2, \nu_2) = (10^{7}, 0.45)$, see~\cref{fig:elas}.
\begin{figure}[htbp]
  \centering
  \begin{tikzpicture}[remember picture]
      \node[inner sep=-0.1cm] (fig) {\includegraphics[width=0.55\textwidth]{Data/elasticity.png}};
  \end{tikzpicture}
      \begin{tikzpicture}[overlay,remember picture]
      \begin{axis}[
              at=(fig.south west),
          xshift=-2.25cm,yshift=-2cm,
    enlargelimits=false,
    colorbar,
    colormap name=paraview,
    point meta min=-4.9e-3,
    point meta max=4.9e-3,
    hide axis,
    colorbar style={
        ytick style={draw=none},
          title={$u(x, \cdot, z)$},
        scaled y ticks = false,
        height = 2.0cm,
        font = \tiny,
        width = 0.3cm,
        ytick={-4.9e-3,0.0,4.9e-3}
    }
    ]
      \end{axis}
  \end{tikzpicture}
    \caption{Deformed geometrical configuration of a 3D clamped beam subject to gravity. The striped plane in the background shows a cut of the resting configuration with the jumps in the coefficient $E_1={\color{paraviewblue}10^7}$ and $E_2={\color{paraviewred}2\times 10^{11}}$, aligned with those of $\nu$.\label{fig:elas}}
\end{figure}
The following solvers are compared:
\begin{enumerate}
    \item \pk{PCGAMG};\label{enum:gamg}
    \item \pk{PCHPDDM} with an exact coarse solver;\label{enum:exact}
    \item \pk{PCHPDDM} with an inexact coarse solver using \pk{PCHYPRE};\label{enum:inexacthypre}
    \item \pk{PCHPDDM} with inexact coarse solvers using GenEO multilevel extension, cf.~\cref{sec:ml}.\label{enum:inexact}
\end{enumerate}
The options -ksp\_gmres\_modifiedgramschmidt
-ksp\_gmres\_restart 50
-ksp\_type fgmres are used in conjunction with all \pk{PC}. The options specific to each solver described above are given next. \\[-4pt]
\begin{minipage}[t]{0.34\textwidth}
\begin{Verbatim}[fontsize=\footnotesize,frame=single,framerule=0.1mm,commandchars=\\\{\}]
\fvtextcolor{mygreen}{# cf. \cref{enum:gamg}}
-pc_type gamg
-prefix_push pc_gamg_
 -threshold 0.03
 -square_graph 4
 -sym_graph true
 -asm_use_agg true
 -repartition true
-prefix_pop
-prefix_push mg_levels_
 -pc_asm_overlap 0
 -sub_pc_type cholesky
-prefix_pop
-prefix_push mg_coarse_
 -pc_type redundant
 -redundant_pc_type cholesky
-prefix_pop
\end{Verbatim}
\end{minipage}
\hfill
\begin{minipage}[t]{0.49\textwidth}
\begin{Verbatim}[fontsize=\footnotesize,frame=single,framerule=0.1mm,commandchars=\\\{\}]
\fvtextcolor{mygreen}{# cf. \cref{enum:exact}}
-pc_type hpddm
-prefix_push pc_hpddm_
 -prefix_push levels_1_
  -pc_type     asm
  -eps_nev     40
  -sub_pc_type cholesky
  -sub_pc_factor_mat_solver_type mumps
  -st_pc_factor_mat_solver_type mumps
 -prefix_pop
 -prefix_push coarse_
  -p 24
  -pc_factor_mat_solver_type mkl_cpardiso
 -prefix_pop
 -define_subdomains
 -has_neumann
-prefix_pop
\end{Verbatim}
\end{minipage} \\
By default, the coarse problem in \pk{PCHPDDM} is solved using an exact $LU$ or Cholesky factorization, in this case using 24 processes and MKL CPARDISO. Switching to an inexact solver using \pk{PCHYPRE} (resp.~GenEO multilevel extension), cf.~\cref{enum:inexacthypre} (resp.~\cref{enum:inexact}), is straightforward. \\[-4pt]
\begin{minipage}[t]{0.4\textwidth}
\begin{Verbatim}[fontsize=\footnotesize,frame=single,framerule=0.1mm,commandchars=\\\{\}, codes={\catcode`$=3\catcode`^=7}]
\fvtextcolor{mygreen}{# cf. \cref{enum:inexacthypre}}
-prefix_push pc_hpddm_coarse_
 -ksp_type    gmres
 -ksp_rtol    1.0e-2
 -ksp_max_it  100
 -ksp_pc_side right
 -pc_type     hypre
 -prefix_push pc_hypre_
  -boomeramg_interp_type  ext+i
  -boomeramg_coarsen_type hmis
 -prefix_pop
-prefix_pop
\end{Verbatim}
\end{minipage}
\hfill
\begin{minipage}[t]{0.525\textwidth}
\begin{Verbatim}[fontsize=\footnotesize,frame=single,framerule=0.1mm,commandchars=\\\{\}, codes={\catcode`$=3\catcode`^=7}]
\fvtextcolor{mygreen}{# cf. \cref{enum:inexact}}
-prefix_push pc_hpddm_levels_2_
 -p           $M$
 -ksp_type    gmres            
 -ksp_rtol    1.0e-2           
 -ksp_pc_side right            
 -pc_type     asm
 -eps_nev     80
 -sub_pc_type cholesky
 -sub_pc_factor_mat_solver_type mkl_pardiso
 -st_pc_factor_mat_solver_type mkl_pardiso
-prefix_pop
\end{Verbatim}
\vspace*{-0.5cm}
\end{minipage} \\
$M$ is the number of subdomains used to define the second-level domain decomposition, which is an aggregation of first-level subdomain.
When it comes to convergence, first note that BoomerAMG from \emph{hypre} has a really hard time reaching the stopping criterion of $10^{-2}$ when solving GenEO coarse problems. This is emphasized in~\cref{fig:hypre-conv}, where the lack of convergence at the coarse level with \pk{PCHYPRE} greatly deteriorates the outer solver convergence, this configuration is thus discarded next and motivates the need for GenEO multilevel extension.
\pgfplotstableread{
it res inner
  0  3.059298477541e+01 nan
  1  3.059296797779e+01 100
  2  3.059296195928e+01 15
  3  3.059296184099e+01 15
  4  3.059295742636e+01 13
  5  3.059295742508e+01 15
  6  3.059294966626e+01 14
  7  3.059293719161e+01 95
  8  3.059292563008e+01 100
  9  3.059286757287e+01 100
 10  3.059276348161e+01 100
 11  3.059259253443e+01 100
 12  3.059200963623e+01 100
 13  3.059116858285e+01 100
 14  3.059008706025e+01 100
 15  3.058899295395e+01 100
 16  3.058765538548e+01 100
 17  3.058718339920e+01 100
 18  3.058690573439e+01 100
 19  3.058631705473e+01 100
 20  3.058529856313e+01 100
 21  3.058458303471e+01 100
 22  3.058441527393e+01 100
 23  3.058404813350e+01 100
 24  3.058350395496e+01 100
 25  3.058297671639e+01 100
 26  3.058267730942e+01 100
 27  3.058267032565e+01 100
 28  3.058260186627e+01 19
 29  3.058233416052e+01 100
 30  3.058177883143e+01 100
 31  3.058107492291e+01 100
 32  3.058053304435e+01 100
 33  3.058021917750e+01 100
 34  3.057991067536e+01 100
 35  3.057984123224e+01 100
 36  3.057981166461e+01 100
 37  3.057969981702e+01 100
 38  3.057951130340e+01 100
 39  3.057909738295e+01 100
 40  3.057879065735e+01 100
 41  3.057844662403e+01 100
 42  3.057813110973e+01 100
 43  3.057805844313e+01 100
 44  3.057748148817e+01 100
 45  3.057716298347e+01 100
 46  3.057695365582e+01 100
 47  3.057687162665e+01 100
 48  3.057685976587e+01 100
 49  3.057676977855e+01 100
 50  3.057666112196e+01 100
 51  3.057666112157e+01 100
 52  3.057666066500e+01 15
 53  3.057666020080e+01 15
 54  3.057666000522e+01 12
 55  3.057666000081e+01 15
 56  3.057665991326e+01 15
 57  3.057665913485e+01 95
 58  3.057665848806e+01 100
 59  3.057665837181e+01 100
 60  3.057665830553e+01 100
 61  3.057665733071e+01 100
 62  3.057665722344e+01 100
 63  3.057665608783e+01 100
 64  3.057664613215e+01 100
 65  3.057661541445e+01 100
 66  3.057659606149e+01 100
 67  3.057657581806e+01 100
 68  3.057648665398e+01 100
 69  3.057648651148e+01 100
 70  3.057647274070e+01 100
 71  3.057646787321e+01 100
 72  3.057646700351e+01 100
 73  3.057646583987e+01 100
 74  3.057640958561e+01 100
 75  3.057640655402e+01 100
 76  3.057640464884e+01 100
 77  3.057634384959e+01 100
 78  3.057630002417e+01 100
 79  3.057628670010e+01 100
 80  3.057628640121e+01 100
 81  3.057628480757e+01 100
 82  3.057625930943e+01 100
 83  3.057619950478e+01 100
 84  3.057615004023e+01 100
 85  3.057605590648e+01 100
 86  3.057603257752e+01 100
 87  3.057575042666e+01 100
 88  3.057537924704e+01 100
 89  3.057516167721e+01 100
 90  3.057445320162e+01 100
 91  3.057434150644e+01 100
 92  3.057433943630e+01 100
 93  3.057431618884e+01 100
 94  3.057419597979e+01 100
 95  3.057412470447e+01 100
 96  3.057409280981e+01 100
 97  3.057407988562e+01 100
 98  3.057397192115e+01 100
 99  3.057365742468e+01 100
100  3.057298339599e+01 100
}\loadedtableHYPRE
\begin{figure}[htbp]
    \centering
        \begin{tikzpicture}[font=\scriptsize]
\begin{axis}[%title=\framebox{{\upcase diffusion 2D, $\mathbb{P}_4$ FE, $\varepsilon = \num[exponent-product=\cdot]{1e-06}$}},
        axis y line*=left,ymajorgrids, unbounded coords=jump, xmin=0,xmax=100,extra y ticks={100}, width=0.5\textwidth,extra y tick labels=\empty, extra y tick style={grid=major}, ylabel = {\# of inner iterations~\ref{pgfplots:inner-hypre}}, xlabel = {Outer iteration index}]
\addplot[last, very thick] table[x=it, y=inner] {\loadedtableHYPRE};\label{pgfplots:inner-hypre}
% \node[rectangle, draw, font=\scriptsize, text width = 4.6cm] at (axis cs:12.9, 18) {\texttt{-hpddm\_master\_p 32 \\ -hpddm\_master\_aggregate\_size 16}};
\end{axis}
\begin{axis}[axis x line=none,ymax=31,ymin=30, axis y line*=right, xmin=0,xmax=100,width=0.5\textwidth, ylabel near ticks=right, ylabel = {Unpreconditioned residual~\ref{pgfplots:res-hypre}}]
\addplot[myred, densely dashed, very thick] table[x=it, y=res] {\loadedtableHYPRE};\label{pgfplots:res-hypre}
\end{axis}
\end{tikzpicture}
    \caption{Convergence history of \pk{PCHPDDM} using \pk{PCHYPRE} on 24 processes
    as a coarse solver. The maximum number of iterations of both the inner coarse solver and the outer solver is set to 100.\label{fig:hypre-conv}}
\end{figure}

Second, when properly parametrized, \cref{enum:exact} and \cref{enum:inexact} yield the same number of outer iterates. In~\cref{fig:gamg-geneo}, the convergence history of \pk{PCGAMG} and \pk{PCHPDDM} are reported. In~\cref{fig:inexact}, the number of inner coarse iterations are reported for various values of $M$. In the case of a two-level method,~\cref{enum:exact}, this number is equal to one and is thus not reported in the figure.
\pgfplotstableread{
    it gamg hpddm
  0 3.859411178402e+00 3.859411178402e+00
  1 3.859411159689e+00 3.859411176864e+00
  2 3.859410249427e+00 3.859411160502e+00
  3 3.859409408276e+00 3.859411009079e+00
  4 3.859405618616e+00 3.859410641267e+00
  5 3.859384473758e+00 3.859409504108e+00
  6 3.859336132196e+00 3.859406284759e+00
  7 3.859180574117e+00 3.859396774066e+00
  8 3.858527456325e+00 3.859371139173e+00
  9 3.855946156750e+00 3.859291203708e+00
 10 3.847882304613e+00 3.859050327998e+00
 11 3.823783202371e+00 3.858349404873e+00
 12 3.761793608189e+00 3.856196410336e+00
 13 3.604276380304e+00 3.850223368267e+00
 14 3.219169630144e+00 3.833574948206e+00
 15 2.524138668209e+00 3.789137446493e+00
 16 1.612332237539e+00 3.677085304126e+00
 17 9.250281017522e-01 3.416685094366e+00
 18 5.244208314872e-01 2.914257430799e+00
 19 2.900205723223e-01 2.201797784976e+00
 20 1.639686082701e-01 1.504764597680e+00
 21 8.999329395027e-02 9.651409800649e-01
 22 4.806932060093e-02 5.991352736022e-01
 23 2.623490189992e-02 3.709375100943e-01
 24 1.420399852677e-02 2.262545029178e-01
 25 7.785203261213e-03 1.386897546734e-01
 26 4.279336819903e-03 8.468744514883e-02
 27 2.392528915920e-03 5.199440557582e-02
 28 1.398509701362e-03 3.167797843106e-02
 29 7.951946917375e-04 1.958093540992e-02
 30 4.302913355532e-04 1.195194560026e-02
 31 2.336327784594e-04 7.371078209494e-03
 32 1.242918661343e-04 4.565136636812e-03
 33 6.466088102124e-05 2.757997749188e-03
 34 3.347704352999e-05 1.722488935062e-03
 35               nan  1.040719253466e-03 
 36               nan  6.471646548950e-04 
 37               nan  3.944692230299e-04 
 38               nan  2.413209464780e-04 
 39               nan  1.481390561258e-04 
 40               nan  9.041135719421e-05 
 41               nan  5.606739576810e-05 
 42               nan  3.445256788047e-05 
}\comparisontable
\pgfplotstableread{
it p-16  p-256  p-1024
 1 26    16     13
 2 9     6      6
 3 10    7      6
 4 8     6      6
 5 9     6      5
 6 5     5      5
 7 8     6      5
 8 9     6      5
 9 7     5      5
10 7     6      5
11 7     6      5
12 8     6      5
13 5     5      5
14 6     6      4
15 6     5      5
16 8     6      5
17 6     5      5
18 6     5      5
19 7     5      4
20 8     6      5
21 8     6      5
22 7     5      4
23 7     6      5
24 6     6      5
25 6     6      5
26 6     5      5
27 6     5      4
28 5     5      5
29 6     5      5
30 6     5      5
31 7     5      5
32 8     6      5
33 7     6      4
34 5     5      5
35 5     5      5
36 6     5      4
37 6     5      4
38 6     5      5
39 7     5      4
40 5     5      5
41 5     5      5
42 6     5      nan
43 5     nan    nan
}\comparisonP
\begin{figure}[htbp]
  \centering
    \subfloat[\parbox{5cm}{Numerical comparison of \pk{PCGAMG} and \pk{PCHPDDM}.\label{fig:gamg-geneo}}]{
\begin{tikzpicture}[font=\scriptsize]
\begin{semilogyaxis}[xmin=0,xmax=50,width=0.46\textwidth,
    legend style={at={(1.05,0.95)},cells={anchor=west}},ymajorgrids,
    xlabel={Iteration number},
    ylabel={Unpreconditioned residual},
    enlarge x limits=0.05,
% yticklabel={
% \pgfmathparse{int(\tick)}
% \ifnum\pgfmathresult=0
% $1$
% \else
% \pgfmathparse{int(\tick/ln(10))}{$10^{\pgfmathresult}$}
% \fi
% }
]
\addplot[very thick,last] table [x=it, y index=1] {\comparisontable};
\addplot[very thick,myred, densely dashed] table [x=it, y index=2]       {\comparisontable};
    \legend{{\pk{PCGAMG}},{\pk{PCHPDDM}}}
% lightgray
\end{semilogyaxis}
\end{tikzpicture}
}
    \subfloat[\parbox{5cm}{\# of inner coarse iterations for various values of $M$, the number of second-level subdomains.}\label{fig:inexact}]{
        \begin{tikzpicture}[font=\scriptsize]
\begin{axis}[%title=\framebox{{\upcase diffusion 2D, $\mathbb{P}_4$ FE, $\varepsilon = \num[exponent-product=\cdot]{1e-06}$}},
        ymajorgrids, unbounded coords=jump, xmin=-1,extra y ticks={100}, width=0.46\textwidth,extra y tick labels=\empty, extra y tick style={grid=major},legend columns=3,legend cell align=left, ylabel = {\# of inner iterations}, xlabel = {Outer iteration index}]
\addlegendimage{empty legend}
\addlegendimage{empty legend}
\addlegendimage{empty legend}
\addplot[last, very thick] table[x=it, y=p-16] {\comparisonP};\label{pgfplots:p-16}
\addplot[myred, densely dashed, very thick] table[x=it, y=p-256] {\comparisonP};\label{pgfplots:p-256}
\addplot[myyellow, densely dotted, very thick] table[x=it, y=p-1024] {\comparisonP};\label{pgfplots:p-1024}

   \addlegendentry{\hspace{-.6cm}-pc\_hpddm\_levels\_2\_p $M$ \hspace*{-2.5cm}}
   \addlegendentry{}
   \addlegendentry{}
   \addlegendentry{16}
   \addlegendentry{256}
    \addlegendentry{\pgfmathprintnumber{1024}}
\end{axis}
        \end{tikzpicture}
    }
    \caption{Convergence histories of \pk{PCGAMG} and \pk{PCHPDDM} for solving the 3D system of linear elasticity. For \pk{PCHPDDM}, a comparison of the number of inner coarse iterations is displayed when using an inexact method, cf.~\cref{enum:inexact}.}
\end{figure}
Note that in this test case, \pk{PCGAMG} is faster than all \pk{PCHPDDM}
alternatives. Its setup (resp.~solution) phase takes 70 (resp.~19) seconds while
for the best parameters of \pk{PCHPDDM}, which is
configuration~\ref{pgfplots:p-256} with 256 second-level subdomains, its
respectively 64 and 36 seconds. Out of these 64 seconds, 31 are spent in SLEPc
\pk{EPSSolve} for solving GenEO at each level but the coarsest. The code used to
produce these results was borrowed from the original paper about the multilevel
extension of GenEO~\cite{aldaas2019multi} and is available
at~\url{https://github.com/prj-/aldaas2019multi}.
    \subsubsection{Liouville--Bratu--Gelfand equation} The strong formulation of this nonlinear problem is given by:
\begin{equation}
    -\nabla \cdot (\kappa \nabla u) - 6.2 \exp^u = 0,\label{eq:bratu}
\end{equation}
where $\kappa$ is a heterogeneous coefficient distribution,
see~\cref{fig:kappa}. This equation may model the temperature distribution in
combustion models. Here, it is merely used to test \pk{PCHPDDM} in the context
of solving successive linearized equations, but there are many other techniques
for solving such a nonlinear equation, such as first decomposing the domain and
then linearizing the equation~\cite{dryja1997nonlinear}.
The physical domain $\Omega$ is the unit cube. It is decomposed in $10{,}272$ subdomains.
\Cref{eq:bratu} is discretized using $\mathbb{P}_2$ finite elements.
The number of unknowns is $\pgfmathprintnumber{217} \times 10^6$, with
approximately $\pgfmathprintnumber{29}$ nonzero coefficients per row.
It is solved using a \pk{SNES}, with \pk{PCHPDDM} being the preconditioner for
solving each linearized system, see~\cref{fig:u}.
\begin{figure}[htbp]
  \centering
  \subfloat[Coefficient distribution.\label{fig:kappa}]{
  \begin{tikzpicture}[remember picture]
      \node[inner sep=-0.1cm] (fig) {\includegraphics[width=0.45\textwidth]{Data/kappa.png}};
  \end{tikzpicture}
  }
    \subfloat[Isosurfaces of the solution.\label{fig:u}]{
  \begin{tikzpicture}[remember picture]
      \node[inner sep=-0.1cm] (fq) {\includegraphics[width=0.45\textwidth]{Data/u.png}};
  \end{tikzpicture}
      \begin{tikzpicture}[overlay,remember picture]
      \begin{axis}[
              at=(fig.south west),
          xshift=-1.5cm,yshift=-2cm,
    enlargelimits=false,
    colorbar,
    colormap name=paraview,
    point meta min=6e-2,
    point meta max=8.5,
    hide axis,
    colorbar style={
        ytick style={draw=none},
        title={$\kappa$},
        scaled y ticks = false,
        height = 2.0cm,
        font = \tiny,
        width = 0.3cm,
        ytick={6e-2,4.0,8.5}
    }
    ]
      \end{axis}
      \begin{axis}[
              at=(fq.south east),
          xshift=0.0cm,yshift=-2cm,
    enlargelimits=false,
    colorbar,
    colormap name=paraview,
    point meta min=-1.9,
    point meta max=1.9,
    hide axis,
    colorbar style={
        ytick style={draw=none},
        title={$u$},
        scaled y ticks = false,
        height = 2.0cm,
        font = \tiny,
        width = 0.3cm,
        ytick={-1.9,0,1.9}
    }
    ]
      \end{axis}
  \end{tikzpicture}
  }
    \caption{Solving the Liouville--Bratu--Gelfand equation using \pk{SNESNEWTONLS} and \pk{PCHPDDM}.}
\end{figure}
The GenEO framework has seldom been used in the context of nonlinear problems, but
here, it is shown that it can solve the linearized equations
from~\cref{eq:bratu} in few iterates.  There are two strategies to define the
\pk{PC} here:
\begin{itemize}
\item supply whenever the Jacobian is being updated in the \pk{SNESSetJacobian}
function new unassembled Neumann matrices at the current linearization point with the previously
explained \pk{PCHPDDMSetAuxiliaryMat} function, this is done automatically with a
\pk{DMPLEX} or a \pk{DMP4EST};
\item by reusing the \pk{PCHPDDM} hierarchy assembled when solving the first
    linearized system, using the \pk{SNESSetLagPreconditioner} option.
\end{itemize}
Since the system is not too stiff here, both strategies lead to the same
convergence history, which is displayed in~\cref{fig:bratu-conv}. The complete set of
options is given in~\cref{fig:bratu-options}.
\pgfplotstableread{
it first                second               third                  fourth
 0 1.199398376742e+01   3.397386070782e-04   1.863266214507e-05     8.796157053238e-07
 1 3.105058690048e+00   3.396971540287e-04   1.863121679230e-05     8.795510768269e-07
 2 7.493003609951e-01   3.394715205399e-04   1.862509132651e-05     8.792972699303e-07
 3 2.066956200353e-01   3.372614259474e-04   1.860041089031e-05     8.783878285049e-07
 4 6.309969159499e-02   3.127230257133e-04   1.838790479520e-05     8.701902993653e-07
 5 1.933161037163e-02   2.148249088439e-04   1.707968608004e-05     8.181649294279e-07
 6 6.550219066971e-03   9.681223676375e-05   1.238892191333e-05     6.183954841666e-07
 7 2.324009861121e-03   3.933958932173e-05   6.179174147298e-06     3.194732043996e-07
 8 8.476764474277e-04   1.558086376310e-05   2.605805785728e-06     1.363806062938e-07
 9 3.138242119678e-04   6.269961516841e-06   1.052999054933e-06     5.522010168719e-08
10 1.183695049547e-04   2.522978646378e-06   4.300851965899e-07     2.257459993685e-08
11                nan   1.024961124607e-06   1.753694476531e-07     9.219016303441e-09
12                nan   4.249085515277e-07   7.338448965970e-08     3.852685693794e-09
13                nan   1.763394120011e-07   3.059024771687e-08     1.609255024084e-09
14                nan   7.432342714299e-08   1.297122798875e-08     6.817901759942e-10
15                nan   3.135242203325e-08   5.496143820621e-09     2.896018108835e-10
16                nan   1.326698994886e-08   2.331975304171e-09     1.226613149417e-10
17                nan   5.639489438404e-09   9.871150757910e-10     5.194784893432e-11
18                nan   2.342602656325e-09   4.069402853378e-10     2.137940700304e-11
19                nan                  nan   1.679412238393e-10     8.828401758238e-12
20                nan                  nan                  nan     3.628457807677e-12
}\comparisonBratu
% 4 SNES Function norm 2.502417058684e-09
\begin{SaveVerbatim}[commandchars=\\\{\}]{VerbEnv}
-snes_type            newtonls
-snes_linesearch_type basic
-ksp_pc_side          right
-pc_type              hpddm
-prefix_push pc_hpddm_
 -prefix_push levels_1_
  -pc_type     asm
  -eps_nev     10
  -sub_pc_type cholesky
  -sub_pc_factor_mat_solver_type mumps
 -prefix_pop
 -coarse_p 128
 -define_subdomains
 -has_neumann
-prefix_pop
\end{SaveVerbatim}
\begin{figure}[htbp]
  \centering
    \subfloat[Convergence histories of the four linearized systems.
    At the top of each curve, for the first iterate, the value of the function
    norm is displayed.\label{fig:bratu-conv}]{
\begin{tikzpicture}[font=\scriptsize]
\begin{semilogyaxis}[clip=false,xmin=0,xmax=68,width=0.49\textwidth,
    legend style={at={(1.05,0.95)},cells={anchor=west}},ymajorgrids,
    xlabel={Iteration number},
    extra x ticks={0, 10, 28, 47, 67},extra x tick labels=\empty,
    extra x tick style={grid=major},
    ylabel={Relative unpreconditioned residual},
    enlarge x limits=0.05,
yticklabel={
\pgfmathparse{int(\tick)}
\ifnum\pgfmathresult=0
$1$
\else
\pgfmathparse{int(\tick/ln(10))}{$10^{\pgfmathresult}$}
\fi
}
]
\addplot[very thick,last] table [x=it, y expr=\thisrowno{1}/1.199398376742e+01] {\comparisonBratu};
\addplot[very thick,last] table [x expr=\thisrowno{0}+10, y expr=\thisrowno{2}/3.397386070782e-04] {\comparisonBratu};
\addplot[very thick,last] table [x expr=\thisrowno{0}+28, y expr=\thisrowno{3}/1.863266214507e-05] {\comparisonBratu};
\addplot[very thick,last] table [x expr=\thisrowno{0}+47, y expr=\thisrowno{4}/8.796157053238e-07] {\comparisonBratu};
\node[font=\tiny,fill=white,inner sep=1pt] at (axis cs:0, 2) {12.0};
\node[font=\tiny,fill=white,inner sep=1pt] at (axis cs:12, 2) {3.4E-4};
\node[font=\tiny,fill=white,inner sep=1pt] at (axis cs:28, 2) {1.9E-5};
\node[font=\tiny,fill=white,inner sep=1pt] at (axis cs:47, 2) {8.8E-7};
\node[font=\tiny,fill=white,inner sep=1pt] at (axis cs:67, 2) {2.5E-9};
\end{semilogyaxis}
\end{tikzpicture}
}
    \subfloat[\parbox{5cm}{\pk{SNES}, \pk{KSP}, and \pk{PC} options for the
    solving~\cref{eq:bratu}}\label{fig:bratu-options}]{
        {\setlength{\fboxrule}{0.1mm}
        \fbox{\BUseVerbatim[fontsize=\footnotesize]{VerbEnv}}
        }
    }
    \caption{Numerical performance of \pk{PCHPDDM} in a nonlinear context for
    solving the Liouville--Bratu--Gelfand equation. Default PETSc tolerances are used:
    $10^{-5}$ decrease of relative residuals for linear solves, $10^{-8}$ decrease
    of function norms for nonlinear solves.}
\end{figure}
    \subsubsection{Using PCHPDDM as a coarse grid solver for PCBDDC\label{sec:bddc}}
A last toy problem is explained to show how one may switch between a multilevel
BDDC solver to a two-level BDDC solver using \pk{PCHPDDM} for solving the
coarse problem. The problem is generated using MFEM~\cite{anderson2019mfem}
definite Maxwell example available at \url{https://github.com/mfem/mfem/blob/master/examples/petsc/ex3p.cpp}.
The strong formulation of the continuous problem is given by:
\begin{align}
	\label{eq:maxwell}
	\begin{split}
        \nabla \times (\nabla \times E) + E	&= 0	\quad \text{in } \Omega,\\
        E \times n					&= (1+16\pi^2)\begin{bmatrix} \sin{4\pi y}&
        \sin{4\pi z}&\sin{4\pi x} \end{bmatrix}^T \text{on } \partial \Omega.
	\end{split}
\end{align}
It is discretized using order 1 N\'ed\'elec finite elements. There is a command
line option -{}-nonoverlapping to activate \pk{PCBDDC}. The following command line
is thus used. \\[-4pt]
\begin{Verbatim}[fontsize=\footnotesize,frame=single,framerule=0.1mm,commandchars=&\[\]]
&fvtextcolor[mygreen][$] &fvtextcolor[myyellow][mpirun] -n 512 ./ex3p -m ../../data/fichera.mesh -f 4 --petscopts rc_ex3p_bddc      &fvtextcolor[myred][\]
  --nonoverlapping
\end{Verbatim}
The option file shows how \pk{PCHPDDM} may be composed into \pk{PCBDDC}. \\[-4pt]
\begin{minipage}[t]{0.565\textwidth}
\begin{Verbatim}[fontsize=\footnotesize,frame=single,framerule=0.1mm,commandchars=&\[\]]
&fvtextcolor[mygreen][$] &fvtextcolor[myyellow][cat] rc_ex3p_bddc
-ksp_type      fgmres
-ksp_norm_type unpreconditioned
-ksp_rtol      1.0e-8

-prefix_push pc_bddc_
 -use_deluxe_scaling
 -levels             1
 -adaptive_threshold 2.0
 -coarsening_ratio   8
 
 -neumann_pc_factor_mat_solver_type   mumps
 -dirichlet_pc_factor_mat_solver_type mumps
&fvtextcolor[mygreen][# continue on the left column]
\end{Verbatim}
\end{minipage}
\begin{minipage}[t]{0.425\textwidth}
\begin{Verbatim}[fontsize=\footnotesize,frame=single,framerule=0.1mm,commandchars=&\[\]]
&fvtextcolor[mygreen][# continued from the right column]
 -prefix_push coarse_
  -ksp_converged_reason
  -ksp_type      gmres
  -ksp_max_it    100
  -ksp_rtol      1.0e-1
  -pc_type       hpddm
  -ksp_norm_type preconditioned
  -prefix_push pc_hpddm_
   -levels_1_pc_type     asm
   -levels_1_eps_nev     10
   -levels_1_sub_pc_type cholesky
   -coarse_pc_type       cholesky
  -prefix_pop
 -prefix_pop
-prefix_pop
\end{Verbatim}
\end{minipage}
Deluxe scaling is used to build an adaptive second level with \pk{PCBDDC}. On this
second level, new subdomains are built by aggregating the constraints
of 8 fine-level subdomains. The BDDC coarse problem composed of $\frac{512}{8} =
64$ subdomains is then solved using \pk{PCHPDDM}, which itself builds another adaptive
level using $10$ local eigenvectors per subdomains. The coarsest level is
aggregated on a single process and solved using an exact Cholesky factorization.
The magnitude of the electric field is plotted~\cref{fig:E}. For the
convergence history of the solver, it is~\cref{fig:BDDC}. The outer solver
reaches the prescribed convergence criterion in 17 iterations~\ref{pgfplots:res}. The number of
inner iterations for solving the BDDC coarse problem is reported as well~\ref{pgfplots:inner}.
\pgfplotstableread{
it res inner
  0 2.362991511979e+02 nan
  1 3.301059014393e+01 3
  2 9.008361295505e+00 4
  3 4.198372502615e+00 7
  4 4.193238438798e+00 8
  5 3.757039752765e+00 7
  6 1.610223090345e+00 23
  7 4.119318170796e-01 5
  8 1.304562389595e-01 7
  9 3.996802619765e-02 6
 10 1.230923272811e-02 6
 11 4.922752304029e-03 6
 12 1.532991950467e-03 6
 13 3.772099626068e-04 6
 14 1.206814967573e-04 6
 15 3.229192953782e-05 6
 16 7.563953270757e-06 6
 17 1.638139681380e-06 6
}\loadedtableBDDC
\begin{figure}[htbp]
  \centering
  \subfloat[Magnitude of the electric field.\label{fig:E}]{
  \begin{tikzpicture}[remember picture]
      \node[inner sep=-0.1cm] (fig) {\includegraphics[width=0.45\textwidth]{Data/fichera.png}};
  \end{tikzpicture}
      \begin{tikzpicture}[overlay,remember picture]
      \begin{axis}[
              at=(fig.south west),
          xshift=-1.25cm,yshift=-2cm,
    enlargelimits=false,
    colorbar,
    colormap name=paraview,
    point meta min=0,
    point meta max=1.4,
    hide axis,
    colorbar style={
        ytick style={draw=none},
        title={$||E||$},
        scaled y ticks = false,
        height = 2.0cm,
        font = \tiny,
        width = 0.3cm,
        ytick={0,0.7,1.4}
    }
    ]
      \end{axis}
  \end{tikzpicture}
  }
    \subfloat[\parbox{5cm}{Convergence history of \pk{PCBDDC} using \pk{PCHPDDM}
    as a coarse solver.\label{fig:BDDC}}]{
        \begin{tikzpicture}[font=\scriptsize]
\begin{axis}[%title=\framebox{{\upcase diffusion 2D, $\mathbb{P}_4$ FE, $\varepsilon = \num[exponent-product=\cdot]{1e-06}$}},
        axis y line*=left,ymajorgrids, unbounded coords=jump, xmin=-1,extra y ticks={100}, width=0.5\textwidth,extra y tick labels=\empty, extra y tick style={grid=major}, ylabel = {\# of inner iterations~\ref{pgfplots:inner}}, xlabel = {Outer iteration index}]
\addplot[last, very thick] table[x=it, y=inner] {\loadedtableBDDC};\label{pgfplots:inner}
% \node[rectangle, draw, font=\scriptsize, text width = 4.6cm] at (axis cs:12.9, 18) {\texttt{-hpddm\_master\_p 32 \\ -hpddm\_master\_aggregate\_size 16}};
\end{axis}
\begin{semilogyaxis}[axis x line=none, axis y line*=right, xmin=-1,width=0.5\textwidth, ylabel near ticks=right, ylabel = {Unpreconditioned residual~\ref{pgfplots:res}}]
\addplot[myred, densely dashed, very thick] table[x=it, y=res] {\loadedtableBDDC};\label{pgfplots:res}
\end{semilogyaxis}
\end{tikzpicture}
  }
    \caption{MFEM example ex3p for solving the Fichera corner problem.}
\end{figure}
\section{Conclusion and perspectives}\label{sec:conclusion}
In this paper, the interface between PETSc and HPDDM is presented. The new
registered Krylov and overlapping Schwarz methods are explained and numerical
examples are provided to show the applicability of these. Solving well-known
problems, various options have been described. Some more unconventional test
cases were commented as well, for example using \pk{MatKAIJ} or \pk{MatIS} matrix types.
Overall, the interface paves the way for having recycling and block Krylov
methods plus robust overlapping Schwarz methods using the GenEO framework in
PETSc.

Concerning PETSc, one may notice that it is now possible to do some code
refactorization for applications that had to deal with multiple right-hand sides.
As a striking example, LOBPCG~\cite{knyazev2001toward} or
CISS~\cite{sakurai2003projection} implementations in SLEPc are
currently\footnote{in version 3.13.2} not using blocking when applying 
spectral transformations, but this is being rewritten to use the new routine
\pk{KSPMatSolve}\footnote{\url{https://gitlab.com/slepc/slepc/-/merge_requests/63}} and should be available in version 3.14.0.

Concerning HPDDM, its integration inside PETSc offers a much-needed flexibility
compared to the standard implementation. Indeed, it is now possible to switch at
runtime many parameters that are determined at compile-time in the standard implementation, e.g.,
subdomain solvers, matrix types, and such.


\section*{Acknowledgments}
The authors would like to thank S. Balay, J. Brown, V. Hapla, M. Knepley, J.
Roman, and B. Smith for reviewing the successive merge requests in PETSc and
SLEPc repositories. This work was granted access to the HPC resources of
TGCC@CEA under the allocations A0070607519 made by GENCI.
% We would like to acknowledge the assistance of volunteers in putting
% together this example manuscript and supplement.

\bibliographystyle{siamplain}
\begin{thebibliography}{10}

\bibitem{adams2004ultra}
{\sc M.~Adams, H.~H. Bayraktar, T.~M. Keaveny, and P.~Papadopoulos}, {\em
  {Ultrascalable Implicit Finite Element Analyses in Solid Mechanics with over
  a Half a Billion Degrees of Freedom}}, in Proceedings of the 2004 ACM/IEEE
  Conference on Supercomputing, SC04, IEEE Computer Society, 2004,
  pp.~\mbox{34:1--34:15}.

\bibitem{aldaas2019multi}
{\sc H.~Al~Daas, L.~Grigori, P.~Jolivet, and P.-H. Tournier}, {\em {A
  Multilevel Schwarz Preconditioner Based on a Hierarchy of Robust Coarse
  Spaces}}, Journal of Scientific Computing,  (2019), p.~submitted for
  publication, \url{https://github.com/prj-/aldaas2019multi}.

\bibitem{amestoy2001fully}
{\sc P.~Amestoy, I.~Duff, J.-Y. L'Excellent, and J.~Koster}, {\em A fully
  asynchronous multifrontal solver using distributed dynamic scheduling}, SIAM
  Journal on Matrix Analysis and Applications, 23 (2001), pp.~15--41,
  \url{http://mumps.enseeiht.fr}.

\bibitem{anderson2019mfem}
{\sc R.~Anderson, J.~Andrej, A.~Barker, J.~Bramwell, J.-S. Camier, J.~Cerveny,
  V.~Dobrev, Y.~Dudouit, A.~Fisher, T.~Kolev, et~al.}, {\em {MFEM: a modular
  finite element methods library}}, arXiv:1911.09220,  (2019),
  \url{http://mfem.org}.

\bibitem{badia2014highly}
{\sc S.~Badia, A.~F. Mart{\'\i}n, and J.~Principe}, {\em A highly scalable
  parallel implementation of balancing domain decomposition by constraints},
  SIAM Journal on Scientific Computing, 36 (2014), pp.~C190--C218.

\bibitem{baker2005technique}
{\sc A.~H. Baker, E.~R. Jessup, and T.~Manteuffel}, {\em {A Technique for
  Accelerating the Convergence of Restarted GMRES}}, SIAM Journal on Matrix
  Analysis and Applications, 26 (2005), pp.~962--984.

\bibitem{balay1997cient}
{\sc S.~Balay, W.~Gropp, L.~Curfman~McInnes, and B.~Smith}, {\em Efficient
  management of parallelism in object-oriented numerical software libraries},
  Modern Software Tools in Scientific Computing,  (1997), pp.~163--202.

\bibitem{bavier2012amesos2}
{\sc E.~Bavier, M.~Hoemmen, S.~Rajamanickam, and H.~Thornquist}, {\em {Amesos2
  and Belos: Direct and Iterative Solvers for Large Sparse Linear Systems}},
  Scientific Programming, 20 (2012), pp.~241--255.

\bibitem{Benzi2011a}
{\sc M.~Benzi, M.~A. Olshanskii, and Z.~Wang}, {\em {Modified augmented
  Lagrangian preconditioners for the incompressible Navier--Stokes equations}},
  International Journal for Numerical Methods in Fluids, 66 (2011),
  pp.~486--508.

\bibitem{brown2012composable}
{\sc J.~Brown, M.~G. Knepley, D.~A. May, L.~C. McInnes, and B.~F. Smith}, {\em
  Composable linear solvers for multiphysics}, in 2012 11th International
  Symposium on Parallel and Distributed Computing, IEEE, 2012, pp.~55--62.

\bibitem{BursteddeWilcoxGhattas11}
{\sc C.~Burstedde, L.~C. Wilcox, and O.~Ghattas}, {\em {\texttt{p4est}}:
  Scalable algorithms for parallel adaptive mesh refinement on forests of
  octrees}, SIAM Journal on Scientific Computing, 33 (2011), pp.~1103--1133,
  \url{http://www.p4est.org}.

\bibitem{butler2020high}
{\sc R.~Butler, T.~Dodwell, A.~Reinarz, A.~Sandhu, R.~Scheichl, and
  L.~Seelinger}, {\em High-performance dune modules for solving large-scale,
  strongly anisotropic elliptic problems with applications to aerospace
  composites}, Computer Physics Communications, 249 (2020).

\bibitem{cai1999restricted}
{\sc X.-C. Cai and M.~Sarkis}, {\em {A restricted additive Schwarz
  preconditioner for general sparse linear systems}}, SIAM Journal on
  Scientific Computing, 21 (1999), pp.~792--797.

\bibitem{calandra2012flexible}
{\sc H.~Calandra, S.~Gratton, J.~Langou, X.~Pinel, and X.~Vasseur}, {\em
  {Flexible Variants of Block Restarted GMRES Methods with Application to
  Geophysics}}, SIAM Journal on Scientific Computing, 34 (2012),
  pp.~A714--A736.

\bibitem{carvalho2011flexible}
{\sc L.~M. Carvalho, S.~Gratton, R.~Lago, and X.~Vasseur}, {\em {A Flexible
  Generalized Conjugate Residual Method With Inner Orthogonalization and
  Deflated Restarting}}, SIAM Journal on Matrix Analysis and Applications, 32
  (2011), pp.~1212--1235.

\bibitem{dohrmann2008domain}
{\sc C.~R. Dohrmann, A.~Klawonn, and O.~B. Widlund}, {\em {Domain decomposition
  for less regular subdomains: Overlapping Schwarz in two dimensions}}, SIAM
  Journal on Numerical Analysis, 46 (2008), pp.~2153--2168.

\bibitem{dohrmann2016bddc}
{\sc C.~R. Dohrmann and O.~B. Widlund}, {\em {A BDDC Algorithm with Deluxe
  Scaling for Three-Dimensional $H(\textnormal{curl})$ Problems}},
  Communications on Pure and Applied Mathematics, 69 (2016), pp.~745--770.

\bibitem{dolean2014ddm}
{\sc V.~Dolean, P.~Jolivet, and F.~Nataf}, {\em An Introduction to Domain
  Decomposition Methods\col\ Algorithms, Theory and Parallel Implementation},
  SIAM, 2015.

\bibitem{dryja1997nonlinear}
{\sc M.~Dryja and W.~Hackbusch}, {\em On the nonlinear domain decomposition
  method}, BIT Numerical Mathematics, 37 (1997), pp.~296--311.

\bibitem{erhel1996restarted}
{\sc J.~Erhel, K.~Burrage, and B.~Pohl}, {\em {Restarted GMRES preconditioned
  by deflation}}, Journal of Computational and Applied Mathematics, 69 (1996),
  pp.~303--318.

\bibitem{falgout2002hypre}
{\sc R.~Falgout and U.~Yang}, {\em \emph{hypre}: a library of high performance
  preconditioners}, Computational Science---ICCS 2002,  (2002), pp.~632--641,
  \url{https://www.llnl.gov/casc/hypre}.

\bibitem{gander2006optimized}
{\sc M.~J. Gander}, {\em Optimized {S}chwarz {M}ethods}, SIAM Journal on
  Numerical Analysis, 44 (2006), pp.~699--731.

\bibitem{gee2006ml}
{\sc M.~Gee, C.~Siefert, J.~Hu, R.~Tuminaro, and M.~Sala}, {\em {ML} 5.0
  smoothed aggregation user's guide}, Tech. Report SAND2006-2649, Sandia
  National Laboratories, 2006, \url{http://trilinos.sandia.gov/packages/ml}.

\bibitem{gutknecht2006block}
{\sc M.~H. Gutknecht}, {\em Block {K}rylov space methods for linear systems
  with multiple right-hand sides: an introduction}, in Modern Mathematical
  Models, Methods and Algorithms for Real World Systems, A.~Siddiqui, I.~Duff,
  and O.~Christensen, eds., 2006, pp.~420--447.

\bibitem{gutknecht2008updating}
{\sc M.~H. Gutknecht and T.~Schmelzer}, {\em {Updating the $QR$ decomposition
  of block tridiagonal and block Hessenberg matrices}}, {Applied Numerical
  Mathematics}, 58 (2008), pp.~871--883.

\bibitem{haferssas2017soras}
{\sc R.~Haferssas, P.~Jolivet, and F.~Nataf}, {\em {An additive Schwarz method
  type theory for Lions's algorithm and a Symmetrized Optimized Restricted
  Additive Schwarz Method}}, Journal on Scientific Computing, 39 (2017),
  pp.~A1345--A1365.

\bibitem{hecht2012new}
{\sc F.~Hecht}, {\em New development in {FreeFem++}}, Journal of Numerical
  Mathematics, 20 (2012), pp.~251--266, \url{http://freefem.org}.

\bibitem{Heinlein:2016:PIT}
{\sc A.~Heinlein, A.~Klawonn, and O.~Rheinbach}, {\em A parallel implementation
  of a two-level overlapping {S}chwarz method with energy-minimizing coarse
  space based on {T}rilinos}, SIAM Journal on Scientific Computing, 38 (2016),
  pp.~C713--C747.

\bibitem{hernandez2005ssf}
{\sc V.~Hernandez, J.~E. Roman, and V.~Vidal}, {\em {SLEP}c: A scalable and
  flexible toolkit for the solution of eigenvalue problems}, ACM Transactions
  on Mathematical Software, 31 (2005), pp.~351--362,
  \url{https://slepc.upv.es}.

\bibitem{heroux2005overview}
{\sc M.~A. Heroux, R.~A. Bartlett, V.~E. Howle, R.~J. Hoekstra, J.~J. Hu, T.~G.
  Kolda, R.~B. Lehoucq, K.~R. Long, R.~P. Pawlowski, E.~T. Phipps, et~al.},
  {\em {An overview of the Trilinos project}}, ACM Transactions on Mathematical
  Software (TOMS), 31 (2005), pp.~397--423, \url{https://trilinos.github.io}.

\bibitem{hestenes1952methods}
{\sc M.~R. Hestenes and E.~Stiefel}, {\em {Methods of Conjugate Gradients for
  Solving Linear Systems}}, Journal of Research of the National Bureau of
  Standards, 49 (1952), pp.~409--436.

\bibitem{ji2017breakdown}
{\sc H.~Ji and Y.~Li}, {\em A breakdown-free block conjugate gradient method},
  BIT Numerical Mathematics, 57 (2017), pp.~379--403.

\bibitem{jolivet2013scalable}
{\sc P.~Jolivet, F.~Hecht, F.~Nataf, and C.~Prud'homme}, {\em {Scalable Domain
  Decomposition Preconditioners for Heterogeneous Elliptic Problems}}, in
  Proceedings of the International Conference on High Performance Computing,
  Networking, Storage and Analysis, SC13, ACM, 2013.

\bibitem{jolivet2016block}
{\sc P.~Jolivet and P.-H. Tournier}, {\em {Block Iterative Methods and
  Recycling for Improved Scalability of Linear Solvers}}, in Proceedings of the
  2016 International Conference for High Performance Computing, Networking,
  Storage and Analysis, SC16, IEEE, 2016.

\bibitem{KALANTZIS2018136}
{\sc V.~Kalantzis, A.~C.~I. Malossi, C.~Bekas, A.~Curioni, E.~Gallopoulos, and
  Y.~Saad}, {\em A scalable iterative dense linear system solver for multiple
  right-hand sides in data analytics}, Parallel Computing, 74 (2018),
  pp.~136--153.

\bibitem{karypis1998fast}
{\sc G.~Karypis and V.~Kumar}, {\em A fast and high quality multilevel scheme
  for partitioning irregular graphs}, SIAM Journal on Scientific Computing, 20
  (1998), pp.~359--392,
  \url{http://glaros.dtc.umn.edu/gkhome/metis/parmetis/overview}.

\bibitem{knepley2009mesh}
{\sc M.~G. Knepley and D.~A. Karpeev}, {\em {Mesh algorithms for PDE with Sieve
  I: Mesh distribution}}, Scientific Programming, 17 (2009), pp.~215--230.

\bibitem{knyazev2001toward}
{\sc A.~V. Knyazev}, {\em {Toward the optimal preconditioned eigensolver:
  Locally optimal block preconditioned conjugate gradient method}}, SIAM
  journal on scientific computing, 23 (2001), pp.~517--541.

\bibitem{lehoucq1998arpack}
{\sc R.~Lehoucq, D.~Sorensen, and C.~Yang}, {\em {ARPACK} users’ guide\col\
  solution of large-scale eigenvalue problems with implicitly restarted Arnoldi
  methods}, vol.~6, Society for Industrial and Applied Mathematics, 1998,
  \url{http://www.caam.rice.edu/software/ARPACK}.

\bibitem{mandel1993balancing}
{\sc J.~Mandel}, {\em Balancing domain decomposition}, Communications in
  Numerical Methods in Engineering, 9 (1993), pp.~233--241.

\bibitem{may2016extreme}
{\sc D.~A. May, P.~Sanan, K.~Rupp, M.~G. Knepley, and B.~F. Smith}, {\em
  {Extreme-scale multigrid components within PETSc}}, in Proceedings of the
  Platform for Advanced Scientific Computing Conference, 2016.

\bibitem{moulin2018al}
{\sc J.~Moulin, P.~Jolivet, and O.~Marquet}, {\em {Augmented Lagrangian
  Preconditioner for Large-Scale Hydrodynamic Stability Analysis}}, Computer
  Methods in Applied Mechanics and Engineering, 351 (2019), pp.~718--743,
  \url{https://github.com/prj-/moulin2019al}.

\bibitem{notay2000flexible}
{\sc Y.~Notay}, {\em Flexible conjugate gradients}, SIAM Journal on Scientific
  Computing, 22 (2000), pp.~1444--1460.

\bibitem{o1980block}
{\sc D.~P. O'Leary}, {\em {The Block Conjugate Gradient Algorithm and Related
  Methods}}, Linear Algebra and its Applications, 29 (1980), pp.~293--322.

\bibitem{parks2006recycling}
{\sc M.~L. Parks, E.~de~Sturler, G.~Mackey, D.~D. Johnson, and S.~Maiti}, {\em
  {Recycling Krylov Subspaces for Sequences of Linear Systems}}, SIAM Journal
  on Scientific Computing, 28 (2006), pp.~1651--1674.

\bibitem{pechstein2017unified}
{\sc C.~Pechstein and C.~R. Dohrmann}, {\em {A unified framework for adaptive
  BDDC}}, Electronic Transactions on Numerical Analysis, 46 (2017),
  pp.~273--336.

\bibitem{saad1993flexible}
{\sc Y.~Saad}, {\em {A Flexible Inner--Outer Preconditioned GMRES Algorithm}},
  SIAM Journal on Scientific Computing, 14 (1993), pp.~461--469.

\bibitem{saad2003iterative}
{\sc Y.~Saad}, {\em {Iterative Methods for Sparse Linear Systems}}, SIAM, 2003.

\bibitem{saad1986gmres}
{\sc Y.~Saad and M.~H. Schultz}, {\em {GMRES: a generalized minimal residual
  algorithm for solving nonsymmetric linear systems}}, SIAM Journal on
  Scientific and Statistical Computing, 7 (1986), pp.~856--869.

\bibitem{sakurai2003projection}
{\sc T.~Sakurai and H.~Sugiura}, {\em A projection method for generalized
  eigenvalue problems using numerical integration}, Journal of Computational
  and Applied Mathematics, 159 (2003), pp.~119--128.

\bibitem{schwarz1870alternating}
{\sc H.~Schwarz}, {\em {\"Uber einen Grenz\"ubergang durch alternierendes
  Verfahren}}, Vierteljahrsschrift der Naturforschenden Gesellschaft in
  Z\"urich, 15 (1870), pp.~272--286.

\bibitem{si2013tet}
{\sc H.~Si}, {\em {TetGen: A Quality Tetrahedral Mesh Generator and 3D Delaunay
  Triangulator}}, Tech. Report~13, 2013,
  \url{http://wias-berlin.de/software/tetgen}.

\bibitem{vsistek2013parallel}
{\sc J.~{\v{S}}{\'\i}stek, J.~Mandel, B.~Soused{\'\i}k, and P.~Burda}, {\em
  Parallel implementation of multilevel {BDDC}}, in Numerical Mathematics and
  Advanced Applications 2011, Springer, 2013, pp.~681--689,
  \url{http://users.math.cas.cz/~sistek/software/bddcml.html}.

\bibitem{smith2004domain}
{\sc B.~F. Smith, P.~Bj{\o}rstad, and B.~Gropp}, {\em Domain decomposition:
  parallel multilevel methods for elliptic partial differential equations},
  Cambridge University Press, 2004.

\bibitem{spillane2011robust}
{\sc N.~Spillane, V.~Dolean, P.~Hauret, F.~Nataf, C.~Pechstein, and
  R.~Scheichl}, {\em A robust two-level domain decomposition preconditioner for
  systems of {PDE}s}, Comptes Rendus Mathématique, 349 (2011), pp.~1255--1259.

\bibitem{spillane2013abstract}
{\sc N.~Spillane, V.~Dolean, P.~Hauret, F.~Nataf, C.~Pechstein, and
  R.~Scheichl}, {\em Abstract robust coarse spaces for systems of {PDEs} via
  generalized eigenproblems in the overlaps}, Numerische Mathematik, 126
  (2013), pp.~741--700.

\bibitem{stathopoulos2002block}
{\sc A.~Stathopoulos and K.~Wu}, {\em {A Block Orthogonalization Procedure with
  Constant Synchronization Requirements}}, SIAM Journal on Scientific
  Computing, 23 (2002), pp.~2165--2182.

\bibitem{Stewart2002}
{\sc G.~W. Stewart}, {\em {A Krylov--Schur Algorithm for Large Eigenproblems}},
  SIAM Journal on Matrix Analysis and Applications, 23 (2002), pp.~601--614.

\bibitem{tang2009comparison}
{\sc J.~Tang, R.~Nabben, C.~Vuik, and Y.~Erlangga}, {\em Comparison of
  two-level preconditioners derived from deflation, domain decomposition and
  multigrid methods}, Journal of Scientific Computing, 39 (2009), pp.~340--370.

\bibitem{tournier2016micro}
{\sc P.-H. Tournier, I.~Aliferis, M.~Bonazzoli, M.~de~Buhan, M.~Darbas,
  V.~Dolean, F.~Hecht, P.~Jolivet, I.~El~Kanfoud, C.~Migliaccio, F.~Nataf,
  C.~Pichot, and S.~Semenov}, {\em {Microwave Tomographic Imaging of
  Cerebrovascular Accidents by Using High-Performance Computing}}, Parallel
  Computing, 85 (2019), pp.~88--97.

\bibitem{van1992bi}
{\sc H.~A. Van~der Vorst}, {\em {Bi-CGSTAB: a fast and smoothly converging
  variant of Bi-CG for the solution of nonsymmetric linear systems}}, SIAM
  Journal on Scientific and Statistical Computing, 13 (1992), pp.~631--644.

\bibitem{wakam2013158}
{\sc D.~N. Wakam and F.~Pacull}, {\em Memory efficient hybrid algebraic solvers
  for linear systems arising from compressible flows}, Computers \& Fluids, 80
  (2013), pp.~158--167.

\bibitem{zampini2016pcbddc}
{\sc S.~Zampini}, {\em {PCBDDC: a class of robust dual--primal methods in
  PETSc}}, SIAM Journal on Scientific Computing, 38 (2016), pp.~S282--S306,
  \url{https://www.mcs.anl.gov/petsc/petsc-master/docs/manualpages/PC/PCBDDC.html}.

\end{thebibliography}
\end{document}
